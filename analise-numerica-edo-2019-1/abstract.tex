\documentclass[12pt]{article}
\usepackage[brazil]{babel}
%\usepackage[latin1]{inputenc}
\usepackage[utf8]{inputenc}
\usepackage{amsmath}
% UTF-8 encoding is recommended by ShareLaTex

\title{Resumo do projeto final da disciplina\\ANEDO 2019.1}

\author{Gil S. M. Neto}
\date{}

\begin{document}

\maketitle

O projeto consistirá em solucionar numericamente o problema dos três corpos, modelando as órbitias de: Sol, Terra, Lua ou Sol, Terra, Marte.\\
O Modelo deverá ser capaz de prever um eclipse no primeiro caso, ou um alinhamento planetário no segundo.

\[
F = \frac{Gm_1m_2}{r^2}
\]
\[
F = ma = m\ddot{r}
\]
Onde $G$ é a constante de gravitação universal, $m_1$ massa do primeiro corpo, $m_2$ massa do segundo corpo, $r$ módulo do vetor posição, nos dá a distância entre os corpos\\
$\ddot{r}$ é segunda derivada em relação ao tempo do vetor posição, nos dando a aceleração e também uma EDO a ser resolvida\\
Será implementado um metódo númerico para a solução do sistema de equações diferenciais e uma pequena formulação teórica do problema.


\end{document}
