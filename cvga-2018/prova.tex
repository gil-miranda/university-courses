\documentclass[12pt,letterpaper]{article}
\usepackage{fullpage}
\usepackage[top=2cm, bottom=4.5cm, left=2.5cm, right=2.5cm]{geometry}
\usepackage{amsmath,amsthm,amsfonts,amssymb,amscd}
\usepackage{lastpage}
\usepackage{enumerate}
\usepackage{fancyhdr}
\usepackage{mathrsfs}
\usepackage{xcolor}
\usepackage{graphicx}
\usepackage{listings}
\usepackage{hyperref}
\usepackage{cancel}

\hypersetup{%
  colorlinks=true,
  linkcolor=blue,
  linkbordercolor={0 0 1}
}

\renewcommand\lstlistingname{Algorithm}
\renewcommand\lstlistlistingname{Algorithms}
\def\lstlistingautorefname{Alg.}
\newtheorem{theorem}{Theorem}[section]
\newtheorem{corollary}{Corollary}[theorem]
\newtheorem{lemma}{Lema}


\lstdefinestyle{Python}{
    language        = Python,
    frame           = lines,
    basicstyle      = \footnotesize,
    keywordstyle    = \color{blue},
    stringstyle     = \color{green},
    commentstyle    = \color{red}\ttfamily
}

\setlength{\parindent}{0.0in}
\setlength{\parskip}{0.05in}

% Edit these as appropriate
\newcommand\course{CVGA 2018}
\newcommand\hwnumber{}                  % <-- homework number
\newcommand\NetIDa{Gil Sales Miranda Neto}           % <-- NetID of person #1
\newcommand\NetIDb{DRE: 118037119}           % <-- NetID of person #2 (Comment this line out for problem sets)

\pagestyle{fancyplain}
\headheight 35pt
\lhead{\NetIDa}
\lhead{\NetIDa\\\NetIDb}                 % <-- Comment this line out for problem sets (make sure you are person #1)
\chead{\textbf{\Large Prova de Repescagem CVGA}}
\rhead{\course \\ \today}
\lfoot{}
\cfoot{}
\rfoot{\small\thepage}
\headsep 1.5em

\begin{document}

\section*{Questão 1}
\subsection*{Dados do problema}
\(O(t)\) tem velocidade constante \(\vec{v}\), ou seja \(\dot{O}(t) = \vec{v}\) e \(\vec{v}(t) = (v,0)\)\\
\(P(t) = (x(t),y(t))\)\\
\subsection*{Solução}

Como \(O(t)\) se move sob o eixo horizontal apenas, temos \(O(t) = (vt,0); v\in \mathbb{R}\).\\
Dessa forma \(\dot{O}(t) = (v,0)\) como queremos.\\
Como o vetor \(P(t)\) tem como origem o ponto \(O(t)\) então este é dado na verdade por:
\[P(t) = O(t) + (x^*(t),y^*(t))\]
Onde as funções que queremos \((x(t),y(t)) = O(t) + (x^*(t),y^*(t))\), e portanto
\begin{cases}
  x (t) = vt+x^* (t) \\
  y (t) = 0+y^* (t)
\end{cases}

O problema agora se reduz a encontrar \(x^*(t)\) e \(y^*(t)\), que podemos pensar como o movimento de \(P(t)\) quando \(O(t)\) é constante, mas este movimento é apenas um círculo, e a parametrização de um circulo \(c(t) = (rcos(t),rsen(t)), r \in \mathbb{R}\).
\\ Da física sabemos que a velocidade angular é a variação do ângulo \(t\), portanto \(c(t) = (rcos(\omega t),rsen(\omega t))\), mas \(c(0) = (r,0)\) e como queremos que \(P(0) = (0,0)\), devemos ter \(c(t) = (rcos(\omega t) - r,rsen(\omega t))\), portanto:\\
\(\begin{cases}
x^*(t) = rcos(\omega t) - r \\
y^*(t) = rsen(\omega t) \\
x(t) = (vt-r) + rcos(\omega t)\\
y(t) = rsen(\omega t)
\end{cases}\)\\
E a parametrização do ponto \(P(t)\) é dada por:
\[ P(t) = (vt-r + rcos(\omega t), rsen(\omega t))\]
Deixo uma animação da parametrização criada no Geogebra, que condiz com a animação do Youtube: \url{https://www.geogebra.org/m/skn6pzzm}

%% Q3
\newpage
\section*{Questão 3}
% Rest of the work...
A demonstração para este exercício consistirá em construir \(\gamma\) de forma a possuir todos os elementos de \(\alpha\) e ir 'inserindo' no conjunto também os elementos de \(\beta\) que sejam L.I. com \(\alpha\), dessa maneira chegará um momento em que não há mais elementos em \(\beta\) que sejam L.I. com \(\alpha\) e então \(\gamma\) será um conjunto gerador de \(\mathbb{V}\) e \(\alpha \subset \gamma \subset \beta \cup \alpha \)

\begin{lemma}
  Seja \(\mathbb{V}\) um espaço vetorial, e \(X = \{x_1, x_2, \dots, x_n \} \in \mathbb{V} \), com \(X\) linearmente independente de modo que \(X \cup \{w\}\) é linearmente dependente, \(\forall w \in \mathbb{V}\). Ou seja, \(X\) é um conjunto L.I. maximal, que possui o máximo de elementos LI de um dado espaço vetorial. Então \(X\) gera \(\mathbb{V}\).
  \begin{proof}[Demonstração]
    Seja \(w \in \mathbb{V}\), então por hipotese \(\{w,x_1,\dots,x_n\}\) é L.D., ou seja \(\exists \, a_0,a_1,\dots,a_n \in \mathbb{R}\) com ao menos um dos \(a_i \neq 0\) de forma que
    \[
    a_0w + a_1x_1 + \dots + a_nx_n = 0
    \]
    De fato, \(a_0 \neq 0\), caso contrário \(a_0 = 0 \implies a_i \neq 0, i \neq 0\) mas isso contradiz o fato de \(X\) ser L.I.\\
    Portanto podemos reescrever
    \[
    w = \frac{-a_1x_1}{a_0} + \dots + \frac{-a_nx_n}{a_0}
    \]
    Ou seja, \(w\) é uma combinação linear dos elementos de \(X\), como estamos assumindo \(\forall w \in \mathbb{V}\), então \(X\) gera \(\mathbb{V}\)
  \end{proof}
\end{lemma}

  Agora partindo para a solução do problema, \(\mathbb{V}\) é gerado por \(\beta\) e \(\alpha\) é L.I., com \(n \leq m\) então existe ao menos um elemento de \(\beta\) que é L.I. com o conjunto \(\alpha\), caso contrário teriamos:  \(a_0v_i + a_1u_1 + \dots + a_nu_n = 0\), com \(a_0 \neq 0\) e poderíamos escrever \(v_i = \frac{-a_1u_1}{a_0} + \dots + \frac{-a_nu_n}{a_0}\), ou seja, \( \beta \) seria combinação linear de \(\alpha\).\\
  Logo \(\alpha \cup \{v_i\}\) é L.I. para algum \(v_i \in \beta\).\\
  Repetimos o processo de encontrar algum \(v_j\) que seja L.I. ao novo conjunto  \(\alpha \cup \{v_i\}, j \neq i\) um número finito de vezes \(k\), de forma que \(m = n + k\), teremos então o conjunto \(\alpha \cup \{v_1, \dots, v_k\}\) L.I. de modo que \(\alpha \cup \{v_1, \dots, v_k\} \cup \{v_i\}\) é L.D. \forall \(v_i \in \beta\).\\
  Seja \(\gamma = \alpha \cup \{v_1, \dots, v_k\} \) e tomando \(\{u_1,\dots, u_n\} = \{\varepsilon_1, \dots, \varepsilon_n\}\) e \(v_1, \dots, v_k = \varepsilon_{n+1}, \dots, \varepsilon_m\), teremos o conjunto \[\gamma = \{\varepsilon_1, \dots, \varepsilon_m\}\].
  Aplicando o Lema 1 a \(\gamma\) temos que \(\gamma\) gera \(\mathbb{V}\) e satisfaz \(\alpha \subset \gamma \subset \beta \cup \alpha \)

% ----------------------------------------------------------
% Questão 4
\newpage
\section*{Questão 4}
Dada a hipérbole \(h\)
\[
\frac{x^2}{a^2} - \frac{y^2}{b^2} = 1
\]

Podemos fazer a seguinte mudança de variaveis:
\begin{cases}
x = au\\
y = bv
\end{cases}

Então teremos:
\begin{align*}
  \frac{x^2}{a^2} - \frac{y^2}{b^2} &= 1\\
  &= \frac{\cancel{a^2}u^2}{\cancel{a^2}} - \frac{\cancel{b^2}v^2}{\cancel{b^2}}\\
  &= u^2 - v^2
\end{align*}
Então a transformação linear \(T_1\) que leva a hipérbole \(h\) na nova hipérbole \(h_1\), e: \(T_1 \cdot (x,y) = (u,v)\)\\
Observando agora que em \(h_1\) temos uma diferença de quadrados, e que em \(xy = 1\) temos um produto de dois termos, podemos então pensar na seguinte mudança de variaveis:
\begin{cases}
u = \frac{1}{2}(p-q)\\
v = \frac{1}{2}(p+q)
\end{cases}\\
Dessa maneira:
\begin{align*}
  u^2 - v^2 &= 1\\
  &= \frac{1}{4}(p-q)^2 -\frac{1}{4}(p+q)^2\\
  &= \frac{1}{4}\left(\cancel{p^2} - 2pq + \cancel{q^2} - \cancel{p^2} - 2pq \cancel{-q^2} \right)\\
  &= pq
\end{align*}
Sendo esta \(pq = 1\) a hipérbole \(h_2\). Então há uma transformação linear \(T_2\) que leva \(h_1\) em \(h_2\), de forma que \(T_2 \cdot (u,v) = (p,q)\)\\
Para encontrar a transformação linear que leva \(h_0\) em \(h_2\) basta:
\begin{align*}
  T_1 \cdot (x,y) &= (u,v) \\
  T_2 \cdot (u,v) &= (p,q) \\
  \text{logo:}\\
  (p,q) &=   T_1 \cdot T_2 \cdot (x,y)
\end{align*}
Sendo \(D = T_1 \cdot T_2\), esta é a matriz procurada. Para saber quais são as matrizes \(T_1, T_2\), podemos fazer a matriz jacobiana da mudança de varíavel:\\
\[
T_1 =
\begin{bmatrix}
  \frac{\partial x}{\partial u} & \frac{\partial x}{\partial v }\\ \\
  \frac{\partial y}{\partial u} & \frac{\partial y}{\partial v }
\end{bmatrix} = \begin{bmatrix}
  a & 0\\ \\
  0 & b
\end{bmatrix}
\]
E a matriz \(T_2\)
\[
T_2 =
\begin{bmatrix}
  \frac{\partial u}{\partial p} & \frac{\partial u}{\partial q }\\ \\
  \frac{\partial v}{\partial p} & \frac{\partial v}{\partial q }
\end{bmatrix} = \begin{bmatrix}
  \frac{1}{2} & -\frac{1}{2}
  \\ \\
  \frac{1}{2} & \frac{1}{2}
\end{bmatrix}
\]
E finalmente \(D = T_1 \cdot T_2\)
\[
D =
\begin{bmatrix}
  a & 0\\ \\
  0 & b
\end{bmatrix} \cdot
\begin{bmatrix}
 \frac{1}{2} & -\frac{1}{2}
 \\ \\
 \frac{1}{2} & \frac{1}{2}
\end{bmatrix}
= \begin{bmatrix}
 \frac{a}{2} & -\frac{a}{2}
 \\ \\
 \frac{b}{2} & \frac{b}{2}
\end{bmatrix}
\]
Verificando:
\begin{align*}
  \begin{bmatrix}
   \frac{a}{2} & -\frac{a}{2}
   \\ \\
   \frac{b}{2} & \frac{b}{2}
 \end{bmatrix} \cdot
 \begin{bmatrix}
 x \\
 \\
 y
 \end{bmatrix} =
 \begin{bmatrix}
 \frac{ax}{2} - \frac{ay}{2} &,  \frac{bx}{2} + \frac{by}{2}
 \end{bmatrix}
\end{align*}
Usando essas novas coordenadas em \(h\)
\begin{align*}
  \frac{\left(\frac{ax}{2} - \frac{ay}{2}\right)^2}{a^2} - \frac{\left(\frac{bx}{2} + \frac{by}{2}\right)^2}{b^2} &= 1\\ \\
  b^2 \left(\frac{a^2x^2}{4} - \frac{2a^2xy}{4} + \frac{a^2 y^2}{4}\right) - a^2\left( \frac{b^2x^2}{4} + \frac{2b^2xy}{4} + \frac{b^2 y^2}{4} \right) &= b^2a^2\\
  \cancel{a^2b^2} xy &= \cancel{a^2b^2}\\
  y &= \frac{1}{x}
\end{align*}
% ----------------------------------------------------------
% Questão 5
\newpage
\section*{Questão 5}
% Item a
\subsection*{Item a}
\begin{lemma}
Seja \(X(t)\) um vetor, se \(\lVert X(t) \rVert = c, \forall t \in \mathbb{R}, c\neq 0\), ou seja, vetor de módulo constante não nulo, então:
\[\langle X(t), X'(t) \rangle = 0\]
\begin{proof}[Demonstração]
\[\langle X(t), X(t) \rangle = c^2\]
Como vale para todo \(t\), podemos derivar
\[ \frac{\mathrm{d}}{\mathrm{d}t}\langle X(t), X(t) \rangle = \frac{\mathrm{d}}{\mathrm{d}t}c^2 = 0 \]
Aplicando a regra da cadeia
\begin{align*}
  \frac{\mathrm{d}}{\mathrm{d}t} \langle X(t), X(t) \rangle &=  \langle X(t)', X(t) \rangle + \langle X(t), X'(t) \rangle \\
  &= 2 \langle X(t), X'(t) \rangle = 0
\end{align*}
e \(2 \langle X(t), X'(t) \rangle = 0 \implies \langle X(t), X'(t) \rangle = 0\)
\end{proof}
\end{lemma}

Seja então \(c(t_0) = (x(t_0), y(t_0), z(t_0))\) a posição da particula no tempo \(t_0\) dada pela função \(c: I \to \mathbb{R}^3\), sabemos que velocidade mede a variação da posição, portanto \(\dot{c}(t_0) = (\dot{x}(t_0), \dot{y}(t_0), \dot{z}(t_0))\) é o vetor velocidade no ponto \(t_0\) que é um vetor tangente a curva descrita por \(c(t)\) no ponto \(t_0\). Como \(\dot{c}(t_0)\) é não nulo, então \(\lvert \dot{c}(t_0) \rvert \neq 0\) , podemos então definir um vetor unitário que é tangente a curva:
\[
T(t_0) = \frac{\dot{c}(t_0)}{\lVert \dot{c}(t_0) \rVert}
\]
Então \(T(t_0)\) é o vetor que nos dá a direção da velocidade.\\
Se olharmos para a variação de \(T(t)\), temos \(T'(t)\) e como \(\lVert T(t) \rVert = 1\), então pela Lema 2 \[\langle T(t), T'(t) \rangle = 0\]
Tomamos então o vetor unitário na direção de \(T'(t_0)\), que continua sendo ortogonal a \(T(t_0)\)\\
\[N(t) = \frac{T'(t)}{\lVert T'(t) \rVert}\]
E este é o vetor na direção Normal.\\
Aceleração é a variação da velocidade, temos então que aceleração é dada por \(\frac{\mathrm{d}}{\mathrm{d}t}\dot{c}(t) = \ddot{c}(t)\).\\
Como temos dois vetores ortogonais \(T(t_0)\) e \(N(t_0)\), podemos decompor as componentes de \(\ddot{c}(t_0)\) nestes vetores, de forma que a aceleração será um combinação linear desses vetores ortogonais.\\
Sendo \(x_T\) a componente aceleração tangencial, e \(x_N\) a componente aceleração normal, ambos escalares, temos então definido o vetor de aceleração normal de \(c\) em \(t_0\): \(\ddot{c}_N(t_0)\)
\begin{align*}
  \ddot{c}(t_0) &= x_T(t_0) T(t_0) + x_N(t_0) N(t_0)\\
  \ddot{c}_T(t_0) &= \frac{\langle \ddot{c}(t_0),T(t_0) \rangle}{\lVert T(t_0) \rVert ^2}T(t_0)  = x_T(t_0) T(t_0)\\
  \ddot{c}_N(t_0) &= \frac{\langle \ddot{c}(t_0),N(t_0) \rangle}{\lVert N(t_0) \rVert ^2}N(t_0)  = x_N(t_0) N(t_0)\\
\end{align*}

% Item begin
\subsection*{Item b}
Vamos olhar para a velocidade e aceleração de \(c_1\), aplicando regra da cadeia e derivando em relação a \(s\)\\
\begin{cases}
  \frac{\mathrm{d}}{\mathrm{d}s}c_1(s) = \frac{\mathrm{d}}{\mathrm{d}s}c(\alpha(s)) = \frac{\mathrm{d}}{\mathrm{d}s}c(\alpha(s))\frac{\mathrm{d}}{\mathrm{d}s}\alpha(s) \\

  \frac{\mathrm{d^2}}{\mathrm{d}s^2}c_1(s) = \frac{\mathrm{d^2}}{\mathrm{d}s^2}c(\alpha(s)) = \frac{\mathrm{d^2}}{\mathrm{d}s^2}c(\alpha(s)) \frac{\mathrm{d}}{\mathrm{d}s}\alpha^2(s) + \frac{\mathrm{d}}{\mathrm{d}s}c(\alpha(s)) \frac{\mathrm{d^2}}{\mathrm{d}s^2}\alpha(s)\\
\end{cases}
\\
\\
Como \(\alpha(s_0) = t_0\), podemos reescrever:
\\
\\
\begin{cases}
  \frac{\mathrm{d}}{\mathrm{d}s}c_1(s_0) = \dot{c}(t_0) \frac{\mathrm{d}}{\mathrm{d}s}\alpha(s_0)\\
  \frac{\mathrm{d^2}}{\mathrm{d}s^2}c_1(s_0) = \ddot{c}(t_0)\frac{\mathrm{d}}{\mathrm{d}s}\alpha^2(s_0) + \dot{c}(t_0)\frac{\mathrm{d^2}}{\mathrm{d}s^2}\alpha(s_0)\\
\end{cases}
\\
\\
Mas \(\frac{\mathrm{d}}{\mathrm{d}s}\alpha(s_0)\) e \(\frac{\mathrm{d^2}}{\mathrm{d}s^2}\alpha(s_0)\) são escalares, podemos então encontrar a aceleração normal para \(c_1(s_0)\)

\begin{align*}
  c''_{1N}(s_0) &= \frac{\langle {\alpha'}^2(s_0)\ddot{c}(t_0) + \alpha''(s_0)\dot{c}(t_0),N(t_0) \rangle}{\lVert N(t_0) \rVert ^2}N(t_0)\\
  &= \frac{\langle {\alpha'}^2(s_0)\ddot{c}(t_0),N(t_0) \rangle + \cancelto{0, \text{pelo Lema 2}}{\langle {\alpha''}(s_0)\dot{c}(t_0),N(t_0) \rangle}}{\lVert N(t_0) \rVert ^2}N(t_0)\\
  &= {\alpha'}^2(s_0)\frac{ \langle \ddot{c}(t_0),N(t_0) \rangle}{\lVert N(t_0) \rVert ^2}N(t_0)\\
  &= {\alpha'}^2(s_0) \ddot{c}_N(t_0)\\
  \text{E a norma da aceleração normal}\\
  \lVert c''_{1N}(s_0) \rVert &= \lvert {\alpha'}^2(s_0)\rvert \cdot \lVert \ddot{c}_N(t_0) \rVert
\end{align*}
E Vamos calcular o módulo da velocidade de \(c_1(s_0)\)
\begin{align*}
  \lVert c'_1(s_0) \rVert &= \lvert {\alpha'}(s_0)\rvert \cdot \lVert \dot{c}_N(t_0) \rVert
\end{align*}
Podemos agora comparar a razão entre norma da aceleração normal e norma da velocidade para \(c_1\) e \(c\)
\begin{align*}
  \frac{\lVert \ddot{c}_N(t_0)\rVert}{{\lVert \dot{c}(t_0) \rVert}^2} &= \frac{\lVert c''_{1N}(s_0)\rVert}{{\lVert c'_1(s_0) \rVert}^2}\\ \\
  \frac{\lVert \ddot{c}_N(t_0)\rVert}{{\lVert \dot{c}(t_0) \rVert}^2} &= \frac{\cancel{\lvert {\alpha'}^2(s_0)\rvert} \cdot \lVert \ddot{c}_N(t_0) \rVert}{\cancel{{\lvert {\alpha'}(s_0)\rvert}^2} \cdot {\lVert \dot{c}(t_0) \rVert}^2} \\ \\
  \frac{\lVert \ddot{c}_N(t_0)\rVert}{{\lVert \dot{c}(t_0) \rVert}^2} &= \frac{\lVert \ddot{c}_N(t_0) \rVert}{{\lVert \dot{c}(t_0) \rVert}^2}
\end{align*}

\end{document}
