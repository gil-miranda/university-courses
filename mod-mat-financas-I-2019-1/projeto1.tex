\documentclass[
	% -- opções da classe memoir --
	11pt,				% tamanho da fonte
	openright,			% capítulos começam em pág ímpar (insere página vazia caso preciso)
	oneside,			% para impressão em recto e verso. Oposto a oneside
	a4paper,			% tamanho do papel.
	% -- opções da classe abntex2 --
	%chapter=TITLE,		% títulos de capítulos convertidos em letras maiúsculas
	%section=TITLE,		% títulos de seções convertidos em letras maiúsculas
	%subsection=TITLE,	% títulos de subseções convertidos em letras maiúsculas
	%subsubsection=TITLE,% títulos de subsubseções convertidos em letras maiúsculas
	% -- opções do pacote babel --
	english,			% idioma adicional para hifenização
	french,				% idioma adicional para hifenização
	spanish,			% idioma adicional para hifenização
	brazil,				% o último idioma é o principal do documento
	]{abntex2}

  \usepackage{lmodern}			% Usa a fonte Latin Modern
  \usepackage[T1]{fontenc}		% Selecao de codigos de fonte.
  \usepackage[utf8]{inputenc}		% Codificacao do documento (conversão automática dos acentos)
  \usepackage{indentfirst}		% Indenta o primeiro parágrafo de cada seção.
  \usepackage{color}				% Controle das cores
  \usepackage{graphicx}			% Inclusão de gráficos
  \usepackage{microtype} 			% para melhorias de justificação
  % ---
  \usepackage{caption}
    \DeclareCaptionLabelFormat{nolabel}{}
    \captionsetup{labelformat=nolabel}

    \usepackage{adjustbox} % Used to constrain images to a maximum size
    \usepackage{xcolor} % Allow colors to be defined
    \usepackage{enumerate} % Needed for markdown enumerations to work
    \usepackage{geometry} % Used to adjust the document margins
    \usepackage{amsmath} % Equations
    \usepackage{amssymb} % Equations
    \usepackage{textcomp} % defines textquotesingle
    % Hack from http://tex.stackexchange.com/a/47451/13684:
    \AtBeginDocument{%
        \def\PYZsq{\textquotesingle}% Upright quotes in Pygmentized code
    }
		\usepackage{booktabs}
    \usepackage{upquote} % Upright quotes for verbatim code
    \usepackage{eurosym} % defines \euro
    \usepackage[mathletters]{ucs} % Extended unicode (utf-8) support
    \usepackage[utf8x]{inputenc} % Allow utf-8 characters in the tex document
    \usepackage{fancyvrb} % verbatim replacement that allows latex
    \usepackage{grffile} % extends the file name processing of package graphics
                         % to support a larger range
    % The hyperref package gives us a pdf with properly built
    % internal navigation ('pdf bookmarks' for the table of contents,
    % internal cross-reference links, web links for URLs, etc.)
    \usepackage{hyperref}
    \usepackage{longtable} % longtable support required by pandoc >1.10
    \usepackage{booktabs}  % table support for pandoc > 1.12.2
    \usepackage[inline]{enumitem} % IRkernel/repr support (it uses the enumerate* environment)
    \usepackage[normalem]{ulem} % ulem is needed to support strikethroughs (\sout)
                                % normalem makes italics be italics, not underlines
    \usepackage{mathrsfs}
    \usepackage{fancyvrb}
    \usepackage{fvextra}

        % Colors for the hyperref package
        \definecolor{urlcolor}{rgb}{0,.145,.698}
        \definecolor{linkcolor}{rgb}{.71,0.21,0.01}
        \definecolor{citecolor}{rgb}{.12,.54,.11}

        % ANSI colors
        \definecolor{ansi-black}{HTML}{3E424D}
        \definecolor{ansi-black-intense}{HTML}{282C36}
        \definecolor{ansi-red}{HTML}{E75C58}
        \definecolor{ansi-red-intense}{HTML}{B22B31}
        \definecolor{ansi-green}{HTML}{00A250}
        \definecolor{ansi-green-intense}{HTML}{007427}
        \definecolor{ansi-yellow}{HTML}{DDB62B}
        \definecolor{ansi-yellow-intense}{HTML}{B27D12}
        \definecolor{ansi-blue}{HTML}{208FFB}
        \definecolor{ansi-blue-intense}{HTML}{0065CA}
        \definecolor{ansi-magenta}{HTML}{D160C4}
        \definecolor{ansi-magenta-intense}{HTML}{A03196}
        \definecolor{ansi-cyan}{HTML}{60C6C8}
        \definecolor{ansi-cyan-intense}{HTML}{258F8F}
        \definecolor{ansi-white}{HTML}{C5C1B4}
        \definecolor{ansi-white-intense}{HTML}{A1A6B2}
        \definecolor{ansi-default-inverse-fg}{HTML}{FFFFFF}
        \definecolor{ansi-default-inverse-bg}{HTML}{000000}

        % Pygments definitions

\makeatletter
\def\PY@reset{\let\PY@it=\relax \let\PY@bf=\relax%
\let\PY@ul=\relax \let\PY@tc=\relax%
\let\PY@bc=\relax \let\PY@ff=\relax}
\def\PY@tok#1{\csname PY@tok@#1\endcsname}
\def\PY@toks#1+{\ifx\relax#1\empty\else%
\PY@tok{#1}\expandafter\PY@toks\fi}
\def\PY@do#1{\PY@bc{\PY@tc{\PY@ul{%
\PY@it{\PY@bf{\PY@ff{#1}}}}}}}
\def\PY#1#2{\PY@reset\PY@toks#1+\relax+\PY@do{#2}}

\expandafter\def\csname PY@tok@w\endcsname{\def\PY@tc##1{\textcolor[rgb]{0.73,0.73,0.73}{##1}}}
\expandafter\def\csname PY@tok@c\endcsname{\let\PY@it=\textit\def\PY@tc##1{\textcolor[rgb]{0.25,0.50,0.50}{##1}}}
\expandafter\def\csname PY@tok@cp\endcsname{\def\PY@tc##1{\textcolor[rgb]{0.74,0.48,0.00}{##1}}}
\expandafter\def\csname PY@tok@k\endcsname{\let\PY@bf=\textbf\def\PY@tc##1{\textcolor[rgb]{0.00,0.50,0.00}{##1}}}
\expandafter\def\csname PY@tok@kp\endcsname{\def\PY@tc##1{\textcolor[rgb]{0.00,0.50,0.00}{##1}}}
\expandafter\def\csname PY@tok@kt\endcsname{\def\PY@tc##1{\textcolor[rgb]{0.69,0.00,0.25}{##1}}}
\expandafter\def\csname PY@tok@o\endcsname{\def\PY@tc##1{\textcolor[rgb]{0.40,0.40,0.40}{##1}}}
\expandafter\def\csname PY@tok@ow\endcsname{\let\PY@bf=\textbf\def\PY@tc##1{\textcolor[rgb]{0.67,0.13,1.00}{##1}}}
\expandafter\def\csname PY@tok@nb\endcsname{\def\PY@tc##1{\textcolor[rgb]{0.00,0.50,0.00}{##1}}}
\expandafter\def\csname PY@tok@nf\endcsname{\def\PY@tc##1{\textcolor[rgb]{0.00,0.00,1.00}{##1}}}
\expandafter\def\csname PY@tok@nc\endcsname{\let\PY@bf=\textbf\def\PY@tc##1{\textcolor[rgb]{0.00,0.00,1.00}{##1}}}
\expandafter\def\csname PY@tok@nn\endcsname{\let\PY@bf=\textbf\def\PY@tc##1{\textcolor[rgb]{0.00,0.00,1.00}{##1}}}
\expandafter\def\csname PY@tok@ne\endcsname{\let\PY@bf=\textbf\def\PY@tc##1{\textcolor[rgb]{0.82,0.25,0.23}{##1}}}
\expandafter\def\csname PY@tok@nv\endcsname{\def\PY@tc##1{\textcolor[rgb]{0.10,0.09,0.49}{##1}}}
\expandafter\def\csname PY@tok@no\endcsname{\def\PY@tc##1{\textcolor[rgb]{0.53,0.00,0.00}{##1}}}
\expandafter\def\csname PY@tok@nl\endcsname{\def\PY@tc##1{\textcolor[rgb]{0.63,0.63,0.00}{##1}}}
\expandafter\def\csname PY@tok@ni\endcsname{\let\PY@bf=\textbf\def\PY@tc##1{\textcolor[rgb]{0.60,0.60,0.60}{##1}}}
\expandafter\def\csname PY@tok@na\endcsname{\def\PY@tc##1{\textcolor[rgb]{0.49,0.56,0.16}{##1}}}
\expandafter\def\csname PY@tok@nt\endcsname{\let\PY@bf=\textbf\def\PY@tc##1{\textcolor[rgb]{0.00,0.50,0.00}{##1}}}
\expandafter\def\csname PY@tok@nd\endcsname{\def\PY@tc##1{\textcolor[rgb]{0.67,0.13,1.00}{##1}}}
\expandafter\def\csname PY@tok@s\endcsname{\def\PY@tc##1{\textcolor[rgb]{0.73,0.13,0.13}{##1}}}
\expandafter\def\csname PY@tok@sd\endcsname{\let\PY@it=\textit\def\PY@tc##1{\textcolor[rgb]{0.73,0.13,0.13}{##1}}}
\expandafter\def\csname PY@tok@si\endcsname{\let\PY@bf=\textbf\def\PY@tc##1{\textcolor[rgb]{0.73,0.40,0.53}{##1}}}
\expandafter\def\csname PY@tok@se\endcsname{\let\PY@bf=\textbf\def\PY@tc##1{\textcolor[rgb]{0.73,0.40,0.13}{##1}}}
\expandafter\def\csname PY@tok@sr\endcsname{\def\PY@tc##1{\textcolor[rgb]{0.73,0.40,0.53}{##1}}}
\expandafter\def\csname PY@tok@ss\endcsname{\def\PY@tc##1{\textcolor[rgb]{0.10,0.09,0.49}{##1}}}
\expandafter\def\csname PY@tok@sx\endcsname{\def\PY@tc##1{\textcolor[rgb]{0.00,0.50,0.00}{##1}}}
\expandafter\def\csname PY@tok@m\endcsname{\def\PY@tc##1{\textcolor[rgb]{0.40,0.40,0.40}{##1}}}
\expandafter\def\csname PY@tok@gh\endcsname{\let\PY@bf=\textbf\def\PY@tc##1{\textcolor[rgb]{0.00,0.00,0.50}{##1}}}
\expandafter\def\csname PY@tok@gu\endcsname{\let\PY@bf=\textbf\def\PY@tc##1{\textcolor[rgb]{0.50,0.00,0.50}{##1}}}
\expandafter\def\csname PY@tok@gd\endcsname{\def\PY@tc##1{\textcolor[rgb]{0.63,0.00,0.00}{##1}}}
\expandafter\def\csname PY@tok@gi\endcsname{\def\PY@tc##1{\textcolor[rgb]{0.00,0.63,0.00}{##1}}}
\expandafter\def\csname PY@tok@gr\endcsname{\def\PY@tc##1{\textcolor[rgb]{1.00,0.00,0.00}{##1}}}
\expandafter\def\csname PY@tok@ge\endcsname{\let\PY@it=\textit}
\expandafter\def\csname PY@tok@gs\endcsname{\let\PY@bf=\textbf}
\expandafter\def\csname PY@tok@gp\endcsname{\let\PY@bf=\textbf\def\PY@tc##1{\textcolor[rgb]{0.00,0.00,0.50}{##1}}}
\expandafter\def\csname PY@tok@go\endcsname{\def\PY@tc##1{\textcolor[rgb]{0.53,0.53,0.53}{##1}}}
\expandafter\def\csname PY@tok@gt\endcsname{\def\PY@tc##1{\textcolor[rgb]{0.00,0.27,0.87}{##1}}}
\expandafter\def\csname PY@tok@err\endcsname{\def\PY@bc##1{\setlength{\fboxsep}{0pt}\fcolorbox[rgb]{1.00,0.00,0.00}{1,1,1}{\strut ##1}}}
\expandafter\def\csname PY@tok@kc\endcsname{\let\PY@bf=\textbf\def\PY@tc##1{\textcolor[rgb]{0.00,0.50,0.00}{##1}}}
\expandafter\def\csname PY@tok@kd\endcsname{\let\PY@bf=\textbf\def\PY@tc##1{\textcolor[rgb]{0.00,0.50,0.00}{##1}}}
\expandafter\def\csname PY@tok@kn\endcsname{\let\PY@bf=\textbf\def\PY@tc##1{\textcolor[rgb]{0.00,0.50,0.00}{##1}}}
\expandafter\def\csname PY@tok@kr\endcsname{\let\PY@bf=\textbf\def\PY@tc##1{\textcolor[rgb]{0.00,0.50,0.00}{##1}}}
\expandafter\def\csname PY@tok@bp\endcsname{\def\PY@tc##1{\textcolor[rgb]{0.00,0.50,0.00}{##1}}}
\expandafter\def\csname PY@tok@fm\endcsname{\def\PY@tc##1{\textcolor[rgb]{0.00,0.00,1.00}{##1}}}
\expandafter\def\csname PY@tok@vc\endcsname{\def\PY@tc##1{\textcolor[rgb]{0.10,0.09,0.49}{##1}}}
\expandafter\def\csname PY@tok@vg\endcsname{\def\PY@tc##1{\textcolor[rgb]{0.10,0.09,0.49}{##1}}}
\expandafter\def\csname PY@tok@vi\endcsname{\def\PY@tc##1{\textcolor[rgb]{0.10,0.09,0.49}{##1}}}
\expandafter\def\csname PY@tok@vm\endcsname{\def\PY@tc##1{\textcolor[rgb]{0.10,0.09,0.49}{##1}}}
\expandafter\def\csname PY@tok@sa\endcsname{\def\PY@tc##1{\textcolor[rgb]{0.73,0.13,0.13}{##1}}}
\expandafter\def\csname PY@tok@sb\endcsname{\def\PY@tc##1{\textcolor[rgb]{0.73,0.13,0.13}{##1}}}
\expandafter\def\csname PY@tok@sc\endcsname{\def\PY@tc##1{\textcolor[rgb]{0.73,0.13,0.13}{##1}}}
\expandafter\def\csname PY@tok@dl\endcsname{\def\PY@tc##1{\textcolor[rgb]{0.73,0.13,0.13}{##1}}}
\expandafter\def\csname PY@tok@s2\endcsname{\def\PY@tc##1{\textcolor[rgb]{0.73,0.13,0.13}{##1}}}
\expandafter\def\csname PY@tok@sh\endcsname{\def\PY@tc##1{\textcolor[rgb]{0.73,0.13,0.13}{##1}}}
\expandafter\def\csname PY@tok@s1\endcsname{\def\PY@tc##1{\textcolor[rgb]{0.73,0.13,0.13}{##1}}}
\expandafter\def\csname PY@tok@mb\endcsname{\def\PY@tc##1{\textcolor[rgb]{0.40,0.40,0.40}{##1}}}
\expandafter\def\csname PY@tok@mf\endcsname{\def\PY@tc##1{\textcolor[rgb]{0.40,0.40,0.40}{##1}}}
\expandafter\def\csname PY@tok@mh\endcsname{\def\PY@tc##1{\textcolor[rgb]{0.40,0.40,0.40}{##1}}}
\expandafter\def\csname PY@tok@mi\endcsname{\def\PY@tc##1{\textcolor[rgb]{0.40,0.40,0.40}{##1}}}
\expandafter\def\csname PY@tok@il\endcsname{\def\PY@tc##1{\textcolor[rgb]{0.40,0.40,0.40}{##1}}}
\expandafter\def\csname PY@tok@mo\endcsname{\def\PY@tc##1{\textcolor[rgb]{0.40,0.40,0.40}{##1}}}
\expandafter\def\csname PY@tok@ch\endcsname{\let\PY@it=\textit\def\PY@tc##1{\textcolor[rgb]{0.25,0.50,0.50}{##1}}}
\expandafter\def\csname PY@tok@cm\endcsname{\let\PY@it=\textit\def\PY@tc##1{\textcolor[rgb]{0.25,0.50,0.50}{##1}}}
\expandafter\def\csname PY@tok@cpf\endcsname{\let\PY@it=\textit\def\PY@tc##1{\textcolor[rgb]{0.25,0.50,0.50}{##1}}}
\expandafter\def\csname PY@tok@c1\endcsname{\let\PY@it=\textit\def\PY@tc##1{\textcolor[rgb]{0.25,0.50,0.50}{##1}}}
\expandafter\def\csname PY@tok@cs\endcsname{\let\PY@it=\textit\def\PY@tc##1{\textcolor[rgb]{0.25,0.50,0.50}{##1}}}

\def\PYZbs{\char`\\}
\def\PYZus{\char`\_}
\def\PYZob{\char`\{}
\def\PYZcb{\char`\}}
\def\PYZca{\char`\^}
\def\PYZam{\char`\&}
\def\PYZlt{\char`\<}
\def\PYZgt{\char`\>}
\def\PYZsh{\char`\#}
\def\PYZpc{\char`\%}
\def\PYZdl{\char`\$}
\def\PYZhy{\char`\-}
\def\PYZsq{\char`\'}
\def\PYZdq{\char`\"}
\def\PYZti{\char`\~}
% for compatibility with earlier versions
\def\PYZat{@}
\def\PYZlb{[}
\def\PYZrb{]}
\makeatother
  % ---
  % Pacotes adicionais, usados no anexo do modelo de folha de identificação
  % ---
  \usepackage{multicol}
  \usepackage{multirow}
  % ---

  % ---
  % Pacotes adicionais, usados apenas no âmbito do Modelo Canônico do abnteX2
  % ---
  \usepackage{lipsum}				% para geração de dummy text
  % ---

  % ---
  % Pacotes de citações
  % ---
  \usepackage[brazilian,hyperpageref]{backref}	 % Paginas com as citações na bibl
  \usepackage[alf]{abntex2cite}	% Citações padrão ABNT

  % ---
  % CONFIGURAÇÕES DE PACOTES
  % ---

  % ---
  % Configurações do pacote backref
  % Usado sem a opção hyperpageref de backref
  \renewcommand{\backrefpagesname}{Citado na(s) página(s):~}
  % Texto padrão antes do número das páginas
  \renewcommand{\backref}{}
  % Define os textos da citação
  \renewcommand*{\backrefalt}[4]{
  	\ifcase #1 %
  		Nenhuma citação no texto.%
  	\or
  		Citado na página #2.%
  	\else
  		Citado #1 vezes nas páginas #2.%
  	\fi}%
  % ---

  % ---
  % Informações de dados para CAPA e FOLHA DE ROSTO
  % ---
  \titulo{Projeto I\\MAE001 - Modelagem Mat. em Finanças I \\O Modelo Binomial}
  \autor{Gil Sales M. Neto \& João Victor de Fonseca}
  \local{Brasil}
  \data{Maio, 2019}
  \instituicao{%
    Universidade Federal do Rio de Janeiro
    \par
    Instituto de Matemática
    \par
    Bacharelado em Matemática Aplicada
		\par
		Prof.: Marco Cabral}
  \tipotrabalho{Relatório técnico}
  % O preambulo deve conter o tipo do trabalho, o objetivo,
  % o nome da instituição e a área de concentração
  \preambulo{}
  % ---

  % ---
  % Configurações de aparência do PDF final

  % alterando o aspecto da cor azul
  \definecolor{blue}{RGB}{41,5,195}

  % informações do PDF
  \makeatletter
  \hypersetup{
       	%pagebackref=true,
  		pdftitle={\@title},
  		pdfauthor={\@author},
      	pdfsubject={\imprimirpreambulo},
  	    pdfcreator={LaTeX with abnTeX2},
  		pdfkeywords={abnt}{latex}{abntex}{abntex2}{relatório técnico},
  		colorlinks=true,       		% false: boxed links; true: colored links
      	linkcolor=blue,          	% color of internal links
      	citecolor=blue,        		% color of links to bibliography
      	filecolor=magenta,      		% color of file links
  		urlcolor=blue,
  		bookmarksdepth=4
  }
  \makeatother
  % ---

  % ---
  % Espaçamentos entre linhas e parágrafos
  % ---

  % O tamanho do parágrafo é dado por:
  \setlength{\parindent}{1.3cm}

  % Controle do espaçamento entre um parágrafo e outro:
  \setlength{\parskip}{0.2cm}  % tente também \onelineskip

  % ---
  % compila o indice
  % ---
  \makeindex
  % ---

  % ----
  % Início do documento
  % ----
\begin{document}

% Seleciona o idioma do documento (conforme pacotes do babel)
%\selectlanguage{english}
\selectlanguage{brazil}

% Retira espaço extra obsoleto entre as frases.
\frenchspacing

% ----------------------------------------------------------
% ELEMENTOS PRÉ-TEXTUAIS
% ----------------------------------------------------------
% \pretextual

% ---
% Capa
% ---
% ---

% ---
% Folha de rosto
% (o * indica que haverá a ficha bibliográfica)
% ---
\imprimirfolhaderosto*
% ---
% ---
% inserir o sumario
% ---

\tableofcontents*
\cleardoublepage

\chapter{Os Algoritmos}
Esta seção tem como objetivo apresentar todos os códigos utilizados nas simulações.
\section{Simulação do modelo binomial}

Utilizamos a linguagem {\color{green}Python3} para a implementação do algoritmo que simula os caminhos da ação

\begin{Verbatim}[commandchars=\\\{\}]
{\color{incolor}In [{\color{incolor}2}]:} \PY{c+c1}{\PYZsh{}\PYZsh{} Função para simular os valores da ação dados os parametros}
        \PY{k}{def} \PY{n+nf}{binomial}\PY{p}{(}\PY{n}{S0}\PY{p}{,} \PY{n}{T}\PY{p}{,} \PY{n}{dt}\PY{p}{,} \PY{n}{u}\PY{p}{,} \PY{n}{d}\PY{p}{,} \PY{n}{p}\PY{p}{)}\PY{p}{:}
            \PY{n}{Si} \PY{o}{=} \PY{n}{S0}
            \PY{n}{S} \PY{o}{=} \PY{p}{[}\PY{p}{]}
            \PY{n}{t} \PY{o}{=} \PY{n}{np}\PY{o}{.}\PY{n}{arange}\PY{p}{(}\PY{l+m+mi}{0}\PY{p}{,}\PY{n}{T}\PY{p}{,}\PY{n}{dt}\PY{p}{)}
            \PY{k}{for} \PY{n}{ti} \PY{o+ow}{in} \PY{n}{t}\PY{p}{:}
                \PY{n}{rnd} \PY{o}{=} \PY{n}{np}\PY{o}{.}\PY{n}{random}\PY{o}{.}\PY{n}{rand}\PY{p}{(}\PY{p}{)}
                \PY{k}{if} \PY{n}{rnd} \PY{o}{\PYZlt{}} \PY{n}{p}\PY{p}{:}
                    \PY{n}{Si} \PY{o}{*}\PY{o}{=} \PY{n}{u}
                \PY{k}{else}\PY{p}{:}
                    \PY{n}{Si} \PY{o}{*}\PY{o}{=} \PY{n}{d}
                \PY{n}{S}\PY{o}{.}\PY{n}{append}\PY{p}{(}\PY{n}{Si}\PY{p}{)}
            \PY{k}{return} \PY{n}{S}
\end{Verbatim}
Esse algoritmo tem depedência do pacote {\color{red}Numpy} do Python, e os gráficos foram plotados com o pacote {\color{red}MatPlotLib.pyplot}
\newpage
\section{Plotando o gráfico da questão 1}
Foram plotados dois gráficos, os quais seguem os códigos\\
\\
\subsection{Plot usual}
\begin{Verbatim}[breaklines=true,commandchars=\\\{\}]
{\color{incolor}In [{\color{incolor}3}]:} \PY{c+c1}{\PYZsh{}\PYZsh{} Definição dos parametros}
        \PY{n}{T} \PY{o}{=} \PY{l+m+mi}{20}
        \PY{n}{dt} \PY{o}{=} \PY{l+m+mf}{0.5}
        \PY{n}{S0} \PY{o}{=} \PY{l+m+mi}{5}
        \PY{n}{u} \PY{o}{=} \PY{l+m+mi}{120}\PY{o}{/}\PY{l+m+mi}{110}
        \PY{n}{d} \PY{o}{=} \PY{l+m+mi}{110}\PY{o}{/}\PY{l+m+mi}{120}
        \PY{n}{p} \PY{o}{=} \PY{l+m+mf}{0.5}

        \PY{c+c1}{\PYZsh{}\PYZsh{} Construção da lista de valores da ação pelo tempo}
        \PY{n}{x} \PY{o}{=} \PY{p}{[}\PY{n}{binomial}\PY{p}{(}\PY{n}{S0}\PY{p}{,} \PY{n}{T}\PY{p}{,} \PY{n}{dt}\PY{p}{,} \PY{n}{u}\PY{p}{,} \PY{n}{d}\PY{p}{,} \PY{n}{p}\PY{p}{)} \PY{k}{for} \PY{n}{i} \PY{o+ow}{in} \PY{n+nb}{range}\PY{p}{(}\PY{l+m+mi}{0}\PY{p}{,}\PY{l+m+mi}{20}\PY{p}{)}\PY{p}{]}

        \PY{c+c1}{\PYZsh{}\PYZsh{} Inserir o valor inicial em cada lista}
        \PY{k}{for} \PY{n}{i} \PY{o+ow}{in} \PY{n+nb}{range}\PY{p}{(}\PY{l+m+mi}{0}\PY{p}{,}\PY{l+m+mi}{20}\PY{p}{)}\PY{p}{:}
            \PY{n}{x}\PY{p}{[}\PY{n}{i}\PY{p}{]}\PY{o}{.}\PY{n}{insert}\PY{p}{(}\PY{l+m+mi}{0}\PY{p}{,}\PY{n}{S0}\PY{p}{)}

        \PY{c+c1}{\PYZsh{}\PYZsh{} Discretização do tempo}
        \PY{n}{ts} \PY{o}{=} \PY{n}{np}\PY{o}{.}\PY{n}{linspace}\PY{p}{(}\PY{l+m+mi}{0}\PY{p}{,}\PY{l+m+mi}{20}\PY{p}{,}\PY{l+m+mi}{41}\PY{p}{)}

        \PY{c+c1}{\PYZsh{}\PYZsh{} Plot do gráfico}
        \PY{n}{plt}\PY{o}{.}\PY{n}{figure}\PY{p}{(}\PY{n}{figsize}\PY{o}{=}\PY{p}{(}\PY{l+m+mi}{20}\PY{p}{,}\PY{l+m+mi}{10}\PY{p}{)}\PY{p}{)}
        \PY{k}{for} \PY{n}{i} \PY{o+ow}{in} \PY{n+nb}{range}\PY{p}{(}\PY{l+m+mi}{0}\PY{p}{,}\PY{l+m+mi}{20}\PY{p}{)}\PY{p}{:}
            \PY{n}{plt}\PY{o}{.}\PY{n}{plot}\PY{p}{(}\PY{n}{ts}\PY{p}{,}\PY{n}{x}\PY{p}{[}\PY{n}{i}\PY{p}{]}\PY{p}{,}\PY{n}{label}\PY{o}{=}\PY{l+s+s2}{\PYZdq{}}\PY{l+s+si}{\PYZob{}0:.5f\PYZcb{}}\PY{l+s+s2}{\PYZdq{}}\PY{o}{.}\PY{n}{format}\PY{p}{(}\PY{n}{x}\PY{p}{[}\PY{n}{i}\PY{p}{]}\PY{p}{[}\PY{o}{\PYZhy{}}\PY{l+m+mi}{1}\PY{p}{]}\PY{p}{)}\PY{p}{)}
        \PY{n}{plt}\PY{o}{.}\PY{n}{title}\PY{p}{(}\PY{l+s+s1}{\PYZsq{}}\PY{l+s+s1}{Simulação com 20 caminhos de valor da ação}\PY{l+s+s1}{\PYZsq{}}\PY{p}{)}
        \PY{n}{plt}\PY{o}{.}\PY{n}{legend}\PY{p}{(}\PY{n}{title}\PY{o}{=}\PY{l+s+s1}{\PYZsq{}}\PY{l+s+s1}{Valor final}\PY{l+s+s1}{\PYZsq{}}\PY{p}{,} \PY{n}{loc}\PY{o}{=}\PY{l+s+s1}{\PYZsq{}}\PY{l+s+s1}{upper center}\PY{l+s+s1}{\PYZsq{}}\PY{p}{,} \PY{n}{fancybox}\PY{o}{=}\PY{k+kc}{True}\PY{p}{,} \PY{n}{shadow}\PY{o}{=}\PY{k+kc}{True}\PY{p}{,}
                   \PY{n}{ncol}\PY{o}{=}\PY{l+m+mi}{7}\PY{p}{,} \PY{n}{bbox\PYZus{}to\PYZus{}anchor}\PY{o}{=}\PY{p}{(}\PY{l+m+mf}{0.5}\PY{p}{,}\PY{l+m+mf}{1.05}\PY{p}{)}\PY{p}{)}
        \PY{n}{plt}\PY{o}{.}\PY{n}{xlabel}\PY{p}{(}\PY{l+s+s1}{\PYZsq{}}\PY{l+s+s1}{Tempo}\PY{l+s+s1}{\PYZsq{}}\PY{p}{)}
        \PY{n}{plt}\PY{o}{.}\PY{n}{ylabel}\PY{p}{(}\PY{l+s+s1}{\PYZsq{}}\PY{l+s+s1}{Valor da Ação.}\PY{l+s+s1}{\PYZsq{}}\PY{p}{)}
        \PY{n}{plt}\PY{o}{.}\PY{n}{show}\PY{p}{(}\PY{p}{)}
\end{Verbatim}
\newpage
\subsection{Plot com escala Log em Y}
\begin{Verbatim}[breaklines=true,commandchars=\\\{\}]
{\color{incolor}In [{\color{incolor}4}]:} \PY{n}{plt}\PY{o}{.}\PY{n}{figure}\PY{p}{(}\PY{n}{figsize}\PY{o}{=}\PY{p}{(}\PY{l+m+mi}{20}\PY{p}{,}\PY{l+m+mi}{10}\PY{p}{)}\PY{p}{)}
        \PY{k}{for} \PY{n}{i} \PY{o+ow}{in} \PY{n+nb}{range}\PY{p}{(}\PY{l+m+mi}{0}\PY{p}{,}\PY{l+m+mi}{20}\PY{p}{)}\PY{p}{:}
            \PY{n}{plt}\PY{o}{.}\PY{n}{semilogy}\PY{p}{(}\PY{n}{ts}\PY{p}{,}\PY{n}{x}\PY{p}{[}\PY{n}{i}\PY{p}{]}\PY{p}{,}\PY{n}{label}\PY{o}{=}\PY{l+s+s2}{\PYZdq{}}\PY{l+s+si}{\PYZob{}0:.5f\PYZcb{}}\PY{l+s+s2}{\PYZdq{}}\PY{o}{.}\PY{n}{format}\PY{p}{(}\PY{n}{x}\PY{p}{[}\PY{n}{i}\PY{p}{]}\PY{p}{[}\PY{o}{\PYZhy{}}\PY{l+m+mi}{1}\PY{p}{]}\PY{p}{)}\PY{p}{)}
        \PY{n}{plt}\PY{o}{.}\PY{n}{legend}\PY{p}{(}\PY{n}{title}\PY{o}{=}\PY{l+s+s1}{\PYZsq{}}\PY{l+s+s1}{Valor final}\PY{l+s+s1}{\PYZsq{}}\PY{p}{,} \PY{n}{loc}\PY{o}{=}\PY{l+s+s1}{\PYZsq{}}\PY{l+s+s1}{upper center}\PY{l+s+s1}{\PYZsq{}}\PY{p}{,} \PY{n}{fancybox}\PY{o}{=}\PY{k+kc}{True}\PY{p}{,} \PY{n}{shadow}\PY{o}{=}\PY{k+kc}{True}\PY{p}{,} \PY{n}{ncol}\PY{o}{=}\PY{l+m+mi}{7}\PY{p}{,} \PY{n}{bbox\PYZus{}to\PYZus{}anchor}\PY{o}{=}\PY{p}{(}\PY{l+m+mf}{0.5}\PY{p}{,}\PY{l+m+mf}{1.05}\PY{p}{)}\PY{p}{)}
        \PY{n}{plt}\PY{o}{.}\PY{n}{title}\PY{p}{(}\PY{l+s+s1}{\PYZsq{}}\PY{l+s+s1}{Caminhos de valor da ação}\PY{l+s+s1}{\PYZsq{}}\PY{p}{)}
        \PY{n}{plt}\PY{o}{.}\PY{n}{show}\PY{p}{(}\PY{p}{)}
\end{Verbatim}

\section{Boxplots da questão 2}
\begin{Verbatim}[commandchars=\\\{\}]
{\color{incolor}In [{\color{incolor}54}]:} \PY{n}{vs} \PY{o}{=} \PY{p}{[}\PY{p}{[}\PY{p}{]}\PY{p}{,}\PY{p}{[}\PY{p}{]}\PY{p}{]}
         \PY{k}{for} \PY{n}{j} \PY{o+ow}{in} \PY{p}{[}\PY{l+m+mi}{250}\PY{p}{,}\PY{l+m+mi}{500}\PY{p}{,}\PY{l+m+mi}{1000}\PY{p}{,}\PY{l+m+mi}{2000}\PY{p}{]}\PY{p}{:}
             \PY{n}{vs}\PY{p}{[}\PY{l+m+mi}{0}\PY{p}{]}\PY{o}{.}\PY{n}{append}\PY{p}{(}\PY{p}{[}\PY{n}{binomial}\PY{p}{(}\PY{n}{S0}\PY{p}{,} \PY{n}{T}\PY{p}{,} \PY{n}{dt}\PY{p}{,} \PY{n}{u}\PY{p}{,} \PY{n}{d}\PY{p}{,} \PY{n}{p}\PY{p}{)}\PY{p}{[}\PY{o}{\PYZhy{}}\PY{l+m+mi}{1}\PY{p}{]} \PY{k}{for} \PY{n}{i} \PY{o+ow}{in} \PY{n+nb}{range}\PY{p}{(}\PY{l+m+mi}{0}\PY{p}{,}\PY{n}{j}\PY{p}{)}\PY{p}{]}\PY{p}{)}
         \PY{n}{plt}\PY{o}{.}\PY{n}{boxplot}\PY{p}{(}\PY{n}{vs}\PY{p}{[}\PY{l+m+mi}{0}\PY{p}{]}\PY{p}{)}
         \PY{n}{plt}\PY{o}{.}\PY{n}{title}\PY{p}{(}\PY{l+s+s1}{\PYZsq{}}\PY{l+s+s1}{Boxplot com simulações}\PY{l+s+s1}{\PYZsq{}}\PY{p}{)}
         \PY{n}{plt}\PY{o}{.}\PY{n}{xlabel}\PY{p}{(}\PY{l+s+s1}{\PYZsq{}}\PY{l+s+s1}{Número de caminhos executados na simulação}\PY{l+s+s1}{\PYZsq{}}\PY{p}{)}
         \PY{n}{plt}\PY{o}{.}\PY{n}{xticks}\PY{p}{(}\PY{p}{[}\PY{l+m+mi}{1}\PY{p}{,}\PY{l+m+mi}{2}\PY{p}{,}\PY{l+m+mi}{3}\PY{p}{,}\PY{l+m+mi}{4}\PY{p}{]}\PY{p}{,}\PY{p}{[}\PY{l+m+mi}{250}\PY{p}{,}\PY{l+m+mi}{500}\PY{p}{,}\PY{l+m+mi}{1000}\PY{p}{,}\PY{l+m+mi}{2000}\PY{p}{]}\PY{p}{)}
         \PY{n}{plt}\PY{o}{.}\PY{n}{show}\PY{p}{(}\PY{p}{)}
\end{Verbatim}
\section{Boxplots da questão 3}
\begin{Verbatim}[commandchars=\\\{\}]
{\color{incolor}In [{\color{incolor}55}]:} \PY{n}{u\PYZus{}n} \PY{o}{=} \PY{n}{np}\PY{o}{.}\PY{n}{sqrt}\PY{p}{(}\PY{n}{u}\PY{p}{)}
         \PY{n}{d\PYZus{}n} \PY{o}{=} \PY{n}{np}\PY{o}{.}\PY{n}{sqrt}\PY{p}{(}\PY{n}{d}\PY{p}{)}
         \PY{n}{dt\PYZus{}n} \PY{o}{=} \PY{n}{dt}\PY{o}{/}\PY{l+m+mi}{2}
         \PY{k}{for} \PY{n}{j} \PY{o+ow}{in} \PY{p}{[}\PY{l+m+mi}{250}\PY{p}{,}\PY{l+m+mi}{500}\PY{p}{,}\PY{l+m+mi}{1000}\PY{p}{,}\PY{l+m+mi}{2000}\PY{p}{]}\PY{p}{:}
             \PY{n}{vs}\PY{p}{[}\PY{l+m+mi}{1}\PY{p}{]}\PY{o}{.}\PY{n}{append}\PY{p}{(}\PY{p}{[}\PY{n}{binomial}\PY{p}{(}\PY{n}{S0}\PY{p}{,} \PY{n}{T}\PY{p}{,} \PY{n}{dt\PYZus{}n}\PY{p}{,} \PY{n}{u\PYZus{}n}\PY{p}{,} \PY{n}{d\PYZus{}n}\PY{p}{,} \PY{n}{p}\PY{p}{)}\PY{p}{[}\PY{o}{\PYZhy{}}\PY{l+m+mi}{1}\PY{p}{]} \PY{k}{for} \PY{n}{i} \PY{o+ow}{in} \PY{n+nb}{range}\PY{p}{(}\PY{l+m+mi}{0}\PY{p}{,}\PY{n}{j}\PY{p}{)}\PY{p}{]}\PY{p}{)}

         \PY{n}{plt}\PY{o}{.}\PY{n}{boxplot}\PY{p}{(}\PY{n}{vs}\PY{p}{[}\PY{l+m+mi}{1}\PY{p}{]}\PY{p}{)}

         \PY{n}{plt}\PY{o}{.}\PY{n}{title}\PY{p}{(}\PY{l+s+s1}{\PYZsq{}}\PY{l+s+s1}{Boxplot com simulações}\PY{l+s+s1}{\PYZsq{}}\PY{p}{)}
         \PY{n}{plt}\PY{o}{.}\PY{n}{ylabel}\PY{p}{(}\PY{l+s+s1}{\PYZsq{}}\PY{l+s+s1}{Valores das ações ao tempo final}\PY{l+s+s1}{\PYZsq{}}\PY{p}{)}
         \PY{n}{plt}\PY{o}{.}\PY{n}{xlabel}\PY{p}{(}\PY{l+s+s1}{\PYZsq{}}\PY{l+s+s1}{Número de caminhos executados na simulação}\PY{l+s+s1}{\PYZsq{}}\PY{p}{)}
         \PY{n}{plt}\PY{o}{.}\PY{n}{xticks}\PY{p}{(}\PY{p}{[}\PY{l+m+mi}{1}\PY{p}{,}\PY{l+m+mi}{2}\PY{p}{,}\PY{l+m+mi}{3}\PY{p}{,}\PY{l+m+mi}{4}\PY{p}{]}\PY{p}{,}\PY{p}{[}\PY{l+m+mi}{250}\PY{p}{,}\PY{l+m+mi}{500}\PY{p}{,}\PY{l+m+mi}{1000}\PY{p}{,}\PY{l+m+mi}{2000}\PY{p}{]}\PY{p}{)}
\end{Verbatim}

\section{Função para calculo da esperança}
\begin{Verbatim}[commandchars=\\\{\}]
{\color{incolor}In [{\color{incolor}56}]:} \PY{c+c1}{\PYZsh{}\PYZsh{} Função para calcular a esperança}
         \PY{k}{def} \PY{n+nf}{esperanca}\PY{p}{(}\PY{n}{p\PYZus{}e}\PY{p}{,}\PY{n}{u\PYZus{}e}\PY{p}{,}\PY{n}{d\PYZus{}e}\PY{p}{,}\PY{n}{S0\PYZus{}e}\PY{p}{,}\PY{n}{n\PYZus{}e}\PY{p}{)}\PY{p}{:}
             \PY{k}{return} \PY{p}{(}\PY{p}{(}\PY{n}{p\PYZus{}e}\PY{o}{*}\PY{n}{u\PYZus{}e}\PY{o}{+}\PY{p}{(}\PY{l+m+mi}{1}\PY{o}{\PYZhy{}}\PY{n}{p\PYZus{}e}\PY{p}{)}\PY{o}{*}\PY{n}{d\PYZus{}e}\PY{p}{)}\PY{o}{*}\PY{o}{*}\PY{n}{n\PYZus{}e}\PY{p}{)}\PY{o}{*}\PY{n}{S0\PYZus{}e}
\end{Verbatim}

\section{Plot dos gráficos de erro}
\begin{Verbatim}[commandchars=\\\{\}]
{\color{incolor}In [{\color{incolor}58}]:} \PY{n}{ts} \PY{o}{=} \PY{n}{np}\PY{o}{.}\PY{n}{linspace}\PY{p}{(}\PY{l+m+mi}{250}\PY{p}{,}\PY{l+m+mi}{2000}\PY{p}{,}\PY{l+m+mi}{4}\PY{p}{)}

         \PY{n}{ys} \PY{o}{=} \PY{p}{[}\PY{p}{[}\PY{p}{]}\PY{p}{,}\PY{p}{[}\PY{p}{]}\PY{p}{]}
         \PY{k}{for} \PY{n}{j} \PY{o+ow}{in} \PY{p}{[}\PY{l+m+mi}{0}\PY{p}{,}\PY{l+m+mi}{1}\PY{p}{]}\PY{p}{:}
             \PY{k}{for} \PY{n}{i} \PY{o+ow}{in} \PY{n}{vs}\PY{p}{[}\PY{n}{j}\PY{p}{]}\PY{p}{:}
                 \PY{n}{ys}\PY{p}{[}\PY{n}{j}\PY{p}{]}\PY{o}{.}\PY{n}{append}\PY{p}{(}\PY{n+nb}{abs}\PY{p}{(}\PY{p}{(}\PY{n}{np}\PY{o}{.}\PY{n}{mean}\PY{p}{(}\PY{n}{i}\PY{p}{)}\PY{o}{\PYZhy{}}\PY{n}{e}\PY{p}{[}\PY{l+m+mi}{0}\PY{p}{]}\PY{p}{)}\PY{p}{)}\PY{p}{)}
         \PY{n}{plt}\PY{o}{.}\PY{n}{plot}\PY{p}{(}\PY{n}{ts}\PY{p}{,}\PY{n}{ys}\PY{p}{[}\PY{l+m+mi}{0}\PY{p}{]}\PY{p}{)}
         \PY{n}{plt}\PY{o}{.}\PY{n}{plot}\PY{p}{(}\PY{n}{ts}\PY{p}{,}\PY{n}{ys}\PY{p}{[}\PY{l+m+mi}{1}\PY{p}{]}\PY{p}{)}
         \PY{n}{plt}\PY{o}{.}\PY{n}{title}\PY{p}{(}\PY{l+s+s1}{\PYZsq{}}\PY{l+s+s1}{Gráfico do erro em relação ao número de simulações}\PY{l+s+s1}{\PYZsq{}}\PY{p}{)}
         \PY{n}{plt}\PY{o}{.}\PY{n}{xlabel}\PY{p}{(}\PY{l+s+s1}{\PYZsq{}}\PY{l+s+s1}{Número de simulações}\PY{l+s+s1}{\PYZsq{}}\PY{p}{)}
         \PY{n}{plt}\PY{o}{.}\PY{n}{ylabel}\PY{p}{(}\PY{l+s+s1}{\PYZsq{}}\PY{l+s+s1}{Erro relativo}\PY{l+s+s1}{\PYZsq{}}\PY{p}{)}
         \PY{n}{plt}\PY{o}{.}\PY{n}{show}\PY{p}{(}\PY{p}{)}
         \PY{n}{plt}\PY{o}{.}\PY{n}{semilogy}\PY{p}{(}\PY{n}{ts}\PY{p}{,}\PY{n}{ys}\PY{p}{[}\PY{l+m+mi}{0}\PY{p}{]}\PY{p}{)}

         \PY{n}{plt}\PY{o}{.}\PY{n}{semilogy}\PY{p}{(}\PY{n}{ts}\PY{p}{,}\PY{n}{ys}\PY{p}{[}\PY{l+m+mi}{1}\PY{p}{]}\PY{p}{)}
         \PY{n}{plt}\PY{o}{.}\PY{n}{title}\PY{p}{(}\PY{l+s+s1}{\PYZsq{}}\PY{l+s+s1}{Gráfico do erro em relação ao número de simulações escala log}\PY{l+s+s1}{\PYZsq{}}\PY{p}{)}
         \PY{n}{plt}\PY{o}{.}\PY{n}{xlabel}\PY{p}{(}\PY{l+s+s1}{\PYZsq{}}\PY{l+s+s1}{Número de simulações}\PY{l+s+s1}{\PYZsq{}}\PY{p}{)}
         \PY{n}{plt}\PY{o}{.}\PY{n}{ylabel}\PY{p}{(}\PY{l+s+s1}{\PYZsq{}}\PY{l+s+s1}{Erro relativo}\PY{l+s+s1}{\PYZsq{}}\PY{p}{)}
         \PY{n}{plt}\PY{o}{.}\PY{n}{show}\PY{p}{(}\PY{p}{)}
\end{Verbatim}



\chapter{Questão 1}
Foi pedido para fixar os parametros \(T = 20\) e \(\Delta t = 0.5\) e escolher os outros, nossos parâmetros então ficaram:
\begin{itemize}
  \item \(T = 20\)
  \item \(\Delta t = 0.5\)
	\item \(S_0 = 5\)
  \item \(u = \frac{120}{110}\)
  \item \(d = \frac{110}{120}\)
  \item \(p = 0.5\)
\end{itemize}
Como estamos tratando de um modelo binomial que lida com exponenciais, faz-se necessário também visualizar os valores com escala log.
\subsection{Plot com escala comum}
  \begin{figure}[h]
    \adjustimage{max size={1.1\linewidth}{1.1\paperheight}}{fig1.png}
  \end{figure}
	\subsection{Plot com escala log}
	\begin{figure}[h]
	  \adjustimage{max size={1.1\linewidth}{1.1\paperheight}}{fig2.png}
	\end{figure}

\chapter{Questão 2}
Foi pedido um boxplot para 250,500,1000,2000 caminhos de valor de ação
\begin{figure}[h]
  \adjustimage{max size={1.5\linewidth}{1.5\paperheight}}{fig3.png}
\end{figure}

\chapter{Questão 3}
Foi pedido um mesmo boxplot, mas com novos parametros, que ficaram:
\begin{itemize}
	\item \(T = 20\)
  \item \(\Delta t = 0.5\)
	\item \(S_0 = 5\)
  \item \(u = \sqrt{\frac{120}{110}}\)
  \item \(d = \sqrt{\frac{110}{120}\)}
  \item \(p = 0.5\)
\end{itemize}
\begin{figure}[h]
  \adjustimage{max size={1.5\linewidth}{1.5\paperheight}}{fig4.png}
\end{figure}

Podemos ver que quando fazemos mais passos com up e down menores, os valores finais ficam bem mais 'comportados', por exemplo os outliers aqui chegaram a no máximo 20, enquanto no caso anterior chegavam a 50.
\chapter{Questão 4}
A média esperada da binomial pode ser dada pela seguinte fórmula:
$$E[X] = (p\cdot u + q\cdot d)^n \cdot S_0$$
onde $q = (1-p), n = T$, e os outros são os parametros do algoritmo\\
A média do conjunto de dados é obtida pela média aritmetica, utilizando o método {\color{red} mean} do numpy.

\subsection{Valor esperado para o primeiro caso}
\begin{align*}
  E[X] &= \left(0.5 \cdot \frac{120}{110} + 0.5 \cdot \frac{110}{120}\right)^{20} \cdot 5\\
  &\approx (1.0038)^{20} \cdot 5\\
  &\approx (1.079) \cdot 5\\
  &\approx 5.393
\end{align*}
\subsection{Valor esperado para o segundo caso}
\begin{align*}
  E[X] &= \left(0.5 \cdot \sqrt{\frac{120}{110}} + 0.5 \cdot \sqrt{\frac{110}{120}}\right)^{20} \cdot 5\\
  &\approx (1.0095)^{20} \cdot 5\\
  &\approx (1.019) \cdot 5\\
  &\approx 5.095
\end{align*}

\subsection{Média dos valores para o primeiro caso}
\begin{tabular}{cc}
	\toprule
	Simulações & Média\\
	\midrule
250 & 5.828 \\
500 & 5.965 \\
1000 & 5.643 \\
2000 & 5.762
\end{tabular}

\subsection{Média dos valores para o segundo caso}
\begin{tabular}{cc}
\toprule
Simulações & Média\\
\midrule
250 & 5.332 \\
500  & 5.342 \\
1000 & 5.326 \\
2000 & 5.415
\end{tabular}

\subsection{Conclusão}
Podemos ver que as médias se aproximam das esperanças a medida que o número de simulações aumenta
\chapter{Questão 5}
\subsection{Plot dos gráficos em escala usual}
A curva azul representa o caso em que \(u = \frac{120}{110}\) e a curva laranja o caso onde \(u = \sqrt{\frac{120}{110}}\)

\begin{figure}[h]
  \adjustimage{max size={1\linewidth}{1\paperheight}}{fig5.png}
\end{figure}
\newpage
\subsection{Plot dos gráficos em escala log}
A curva azul representa o caso em que \(u = \frac{120}{110}\) e a curva laranja o caso onde \(u = \sqrt{\frac{120}{110}}\)

\begin{figure}[h]
  \adjustimage{max size={1\linewidth}{1\paperheight}}{fig6.png}
\end{figure}
\end{document}
