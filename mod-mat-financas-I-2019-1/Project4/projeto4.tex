\documentclass[
	% -- opções da classe memoir --
	11pt,				% tamanho da fonte
	openright,			% capítulos começam em pág ímpar (insere página vazia caso preciso)
	oneside,			% para impressão em recto e verso. Oposto a oneside
	a4paper,			% tamanho do papel.
	% -- opções da classe abntex2 --
	%chapter=TITLE,		% títulos de capítulos convertidos em letras maiúsculas
	%section=TITLE,		% títulos de seções convertidos em letras maiúsculas
	%subsection=TITLE,	% títulos de subseções convertidos em letras maiúsculas
	%subsubsection=TITLE,% títulos de subsubseções convertidos em letras maiúsculas
	% -- opções do pacote babel --
	english,			% idioma adicional para hifenização
	french,				% idioma adicional para hifenização
	spanish,			% idioma adicional para hifenização
	brazil,				% o último idioma é o principal do documento
	]{abntex2}

  \usepackage{lmodern}			% Usa a fonte Latin Modern
  \usepackage[T1]{fontenc}		% Selecao de codigos de fonte.
  \usepackage[utf8]{inputenc}		% Codificacao do documento (conversão automática dos acentos)
  \usepackage{indentfirst}		% Indenta o primeiro parágrafo de cada seção.
  \usepackage{color}				% Controle das cores
  \usepackage{graphicx}			% Inclusão de gráficos
  \usepackage{microtype} 			% para melhorias de justificação
  % ---
  \usepackage{caption}
    \DeclareCaptionLabelFormat{nolabel}{}
    \captionsetup{labelformat=nolabel}

    \usepackage{adjustbox} % Used to constrain images to a maximum size
    \usepackage{xcolor} % Allow colors to be defined
    \usepackage{enumerate} % Needed for markdown enumerations to work
    \usepackage{geometry} % Used to adjust the document margins
    \usepackage{amsmath} % Equations
    \usepackage{amssymb} % Equations
    \usepackage{textcomp} % defines textquotesingle
    % Hack from http://tex.stackexchange.com/a/47451/13684:
    \AtBeginDocument{%
        \def\PYZsq{\textquotesingle}% Upright quotes in Pygmentized code
    }
		\usepackage{booktabs}
    \usepackage{upquote} % Upright quotes for verbatim code
    \usepackage{eurosym} % defines \euro
    \usepackage[mathletters]{ucs} % Extended unicode (utf-8) support
    \usepackage[utf8x]{inputenc} % Allow utf-8 characters in the tex document
    \usepackage{fancyvrb} % verbatim replacement that allows latex
    \usepackage{grffile} % extends the file name processing of package graphics
                         % to support a larger range
    % The hyperref package gives us a pdf with properly built
    % internal navigation ('pdf bookmarks' for the table of contents,
    % internal cross-reference links, web links for URLs, etc.)
    \usepackage{hyperref}
    \usepackage{longtable} % longtable support required by pandoc >1.10
    \usepackage{booktabs}  % table support for pandoc > 1.12.2
    \usepackage[inline]{enumitem} % IRkernel/repr support (it uses the enumerate* environment)
    \usepackage[normalem]{ulem} % ulem is needed to support strikethroughs (\sout)
                                % normalem makes italics be italics, not underlines
    \usepackage{mathrsfs}
    \usepackage{fancyvrb}
    \usepackage{fvextra}

        % Colors for the hyperref package
        \definecolor{urlcolor}{rgb}{0,.145,.698}
        \definecolor{linkcolor}{rgb}{.71,0.21,0.01}
        \definecolor{citecolor}{rgb}{.12,.54,.11}

        % ANSI colors
        \definecolor{ansi-black}{HTML}{3E424D}
        \definecolor{ansi-black-intense}{HTML}{282C36}
        \definecolor{ansi-red}{HTML}{E75C58}
        \definecolor{ansi-red-intense}{HTML}{B22B31}
        \definecolor{ansi-green}{HTML}{00A250}
        \definecolor{ansi-green-intense}{HTML}{007427}
        \definecolor{ansi-yellow}{HTML}{DDB62B}
        \definecolor{ansi-yellow-intense}{HTML}{B27D12}
        \definecolor{ansi-blue}{HTML}{208FFB}
        \definecolor{ansi-blue-intense}{HTML}{0065CA}
        \definecolor{ansi-magenta}{HTML}{D160C4}
        \definecolor{ansi-magenta-intense}{HTML}{A03196}
        \definecolor{ansi-cyan}{HTML}{60C6C8}
        \definecolor{ansi-cyan-intense}{HTML}{258F8F}
        \definecolor{ansi-white}{HTML}{C5C1B4}
        \definecolor{ansi-white-intense}{HTML}{A1A6B2}
        \definecolor{ansi-default-inverse-fg}{HTML}{FFFFFF}
        \definecolor{ansi-default-inverse-bg}{HTML}{000000}

        % Pygments definitions

\makeatletter
\def\PY@reset{\let\PY@it=\relax \let\PY@bf=\relax%
\let\PY@ul=\relax \let\PY@tc=\relax%
\let\PY@bc=\relax \let\PY@ff=\relax}
\def\PY@tok#1{\csname PY@tok@#1\endcsname}
\def\PY@toks#1+{\ifx\relax#1\empty\else%
\PY@tok{#1}\expandafter\PY@toks\fi}
\def\PY@do#1{\PY@bc{\PY@tc{\PY@ul{%
\PY@it{\PY@bf{\PY@ff{#1}}}}}}}
\def\PY#1#2{\PY@reset\PY@toks#1+\relax+\PY@do{#2}}

\expandafter\def\csname PY@tok@w\endcsname{\def\PY@tc##1{\textcolor[rgb]{0.73,0.73,0.73}{##1}}}
\expandafter\def\csname PY@tok@c\endcsname{\let\PY@it=\textit\def\PY@tc##1{\textcolor[rgb]{0.25,0.50,0.50}{##1}}}
\expandafter\def\csname PY@tok@cp\endcsname{\def\PY@tc##1{\textcolor[rgb]{0.74,0.48,0.00}{##1}}}
\expandafter\def\csname PY@tok@k\endcsname{\let\PY@bf=\textbf\def\PY@tc##1{\textcolor[rgb]{0.00,0.50,0.00}{##1}}}
\expandafter\def\csname PY@tok@kp\endcsname{\def\PY@tc##1{\textcolor[rgb]{0.00,0.50,0.00}{##1}}}
\expandafter\def\csname PY@tok@kt\endcsname{\def\PY@tc##1{\textcolor[rgb]{0.69,0.00,0.25}{##1}}}
\expandafter\def\csname PY@tok@o\endcsname{\def\PY@tc##1{\textcolor[rgb]{0.40,0.40,0.40}{##1}}}
\expandafter\def\csname PY@tok@ow\endcsname{\let\PY@bf=\textbf\def\PY@tc##1{\textcolor[rgb]{0.67,0.13,1.00}{##1}}}
\expandafter\def\csname PY@tok@nb\endcsname{\def\PY@tc##1{\textcolor[rgb]{0.00,0.50,0.00}{##1}}}
\expandafter\def\csname PY@tok@nf\endcsname{\def\PY@tc##1{\textcolor[rgb]{0.00,0.00,1.00}{##1}}}
\expandafter\def\csname PY@tok@nc\endcsname{\let\PY@bf=\textbf\def\PY@tc##1{\textcolor[rgb]{0.00,0.00,1.00}{##1}}}
\expandafter\def\csname PY@tok@nn\endcsname{\let\PY@bf=\textbf\def\PY@tc##1{\textcolor[rgb]{0.00,0.00,1.00}{##1}}}
\expandafter\def\csname PY@tok@ne\endcsname{\let\PY@bf=\textbf\def\PY@tc##1{\textcolor[rgb]{0.82,0.25,0.23}{##1}}}
\expandafter\def\csname PY@tok@nv\endcsname{\def\PY@tc##1{\textcolor[rgb]{0.10,0.09,0.49}{##1}}}
\expandafter\def\csname PY@tok@no\endcsname{\def\PY@tc##1{\textcolor[rgb]{0.53,0.00,0.00}{##1}}}
\expandafter\def\csname PY@tok@nl\endcsname{\def\PY@tc##1{\textcolor[rgb]{0.63,0.63,0.00}{##1}}}
\expandafter\def\csname PY@tok@ni\endcsname{\let\PY@bf=\textbf\def\PY@tc##1{\textcolor[rgb]{0.60,0.60,0.60}{##1}}}
\expandafter\def\csname PY@tok@na\endcsname{\def\PY@tc##1{\textcolor[rgb]{0.49,0.56,0.16}{##1}}}
\expandafter\def\csname PY@tok@nt\endcsname{\let\PY@bf=\textbf\def\PY@tc##1{\textcolor[rgb]{0.00,0.50,0.00}{##1}}}
\expandafter\def\csname PY@tok@nd\endcsname{\def\PY@tc##1{\textcolor[rgb]{0.67,0.13,1.00}{##1}}}
\expandafter\def\csname PY@tok@s\endcsname{\def\PY@tc##1{\textcolor[rgb]{0.73,0.13,0.13}{##1}}}
\expandafter\def\csname PY@tok@sd\endcsname{\let\PY@it=\textit\def\PY@tc##1{\textcolor[rgb]{0.73,0.13,0.13}{##1}}}
\expandafter\def\csname PY@tok@si\endcsname{\let\PY@bf=\textbf\def\PY@tc##1{\textcolor[rgb]{0.73,0.40,0.53}{##1}}}
\expandafter\def\csname PY@tok@se\endcsname{\let\PY@bf=\textbf\def\PY@tc##1{\textcolor[rgb]{0.73,0.40,0.13}{##1}}}
\expandafter\def\csname PY@tok@sr\endcsname{\def\PY@tc##1{\textcolor[rgb]{0.73,0.40,0.53}{##1}}}
\expandafter\def\csname PY@tok@ss\endcsname{\def\PY@tc##1{\textcolor[rgb]{0.10,0.09,0.49}{##1}}}
\expandafter\def\csname PY@tok@sx\endcsname{\def\PY@tc##1{\textcolor[rgb]{0.00,0.50,0.00}{##1}}}
\expandafter\def\csname PY@tok@m\endcsname{\def\PY@tc##1{\textcolor[rgb]{0.40,0.40,0.40}{##1}}}
\expandafter\def\csname PY@tok@gh\endcsname{\let\PY@bf=\textbf\def\PY@tc##1{\textcolor[rgb]{0.00,0.00,0.50}{##1}}}
\expandafter\def\csname PY@tok@gu\endcsname{\let\PY@bf=\textbf\def\PY@tc##1{\textcolor[rgb]{0.50,0.00,0.50}{##1}}}
\expandafter\def\csname PY@tok@gd\endcsname{\def\PY@tc##1{\textcolor[rgb]{0.63,0.00,0.00}{##1}}}
\expandafter\def\csname PY@tok@gi\endcsname{\def\PY@tc##1{\textcolor[rgb]{0.00,0.63,0.00}{##1}}}
\expandafter\def\csname PY@tok@gr\endcsname{\def\PY@tc##1{\textcolor[rgb]{1.00,0.00,0.00}{##1}}}
\expandafter\def\csname PY@tok@ge\endcsname{\let\PY@it=\textit}
\expandafter\def\csname PY@tok@gs\endcsname{\let\PY@bf=\textbf}
\expandafter\def\csname PY@tok@gp\endcsname{\let\PY@bf=\textbf\def\PY@tc##1{\textcolor[rgb]{0.00,0.00,0.50}{##1}}}
\expandafter\def\csname PY@tok@go\endcsname{\def\PY@tc##1{\textcolor[rgb]{0.53,0.53,0.53}{##1}}}
\expandafter\def\csname PY@tok@gt\endcsname{\def\PY@tc##1{\textcolor[rgb]{0.00,0.27,0.87}{##1}}}
\expandafter\def\csname PY@tok@err\endcsname{\def\PY@bc##1{\setlength{\fboxsep}{0pt}\fcolorbox[rgb]{1.00,0.00,0.00}{1,1,1}{\strut ##1}}}
\expandafter\def\csname PY@tok@kc\endcsname{\let\PY@bf=\textbf\def\PY@tc##1{\textcolor[rgb]{0.00,0.50,0.00}{##1}}}
\expandafter\def\csname PY@tok@kd\endcsname{\let\PY@bf=\textbf\def\PY@tc##1{\textcolor[rgb]{0.00,0.50,0.00}{##1}}}
\expandafter\def\csname PY@tok@kn\endcsname{\let\PY@bf=\textbf\def\PY@tc##1{\textcolor[rgb]{0.00,0.50,0.00}{##1}}}
\expandafter\def\csname PY@tok@kr\endcsname{\let\PY@bf=\textbf\def\PY@tc##1{\textcolor[rgb]{0.00,0.50,0.00}{##1}}}
\expandafter\def\csname PY@tok@bp\endcsname{\def\PY@tc##1{\textcolor[rgb]{0.00,0.50,0.00}{##1}}}
\expandafter\def\csname PY@tok@fm\endcsname{\def\PY@tc##1{\textcolor[rgb]{0.00,0.00,1.00}{##1}}}
\expandafter\def\csname PY@tok@vc\endcsname{\def\PY@tc##1{\textcolor[rgb]{0.10,0.09,0.49}{##1}}}
\expandafter\def\csname PY@tok@vg\endcsname{\def\PY@tc##1{\textcolor[rgb]{0.10,0.09,0.49}{##1}}}
\expandafter\def\csname PY@tok@vi\endcsname{\def\PY@tc##1{\textcolor[rgb]{0.10,0.09,0.49}{##1}}}
\expandafter\def\csname PY@tok@vm\endcsname{\def\PY@tc##1{\textcolor[rgb]{0.10,0.09,0.49}{##1}}}
\expandafter\def\csname PY@tok@sa\endcsname{\def\PY@tc##1{\textcolor[rgb]{0.73,0.13,0.13}{##1}}}
\expandafter\def\csname PY@tok@sb\endcsname{\def\PY@tc##1{\textcolor[rgb]{0.73,0.13,0.13}{##1}}}
\expandafter\def\csname PY@tok@sc\endcsname{\def\PY@tc##1{\textcolor[rgb]{0.73,0.13,0.13}{##1}}}
\expandafter\def\csname PY@tok@dl\endcsname{\def\PY@tc##1{\textcolor[rgb]{0.73,0.13,0.13}{##1}}}
\expandafter\def\csname PY@tok@s2\endcsname{\def\PY@tc##1{\textcolor[rgb]{0.73,0.13,0.13}{##1}}}
\expandafter\def\csname PY@tok@sh\endcsname{\def\PY@tc##1{\textcolor[rgb]{0.73,0.13,0.13}{##1}}}
\expandafter\def\csname PY@tok@s1\endcsname{\def\PY@tc##1{\textcolor[rgb]{0.73,0.13,0.13}{##1}}}
\expandafter\def\csname PY@tok@mb\endcsname{\def\PY@tc##1{\textcolor[rgb]{0.40,0.40,0.40}{##1}}}
\expandafter\def\csname PY@tok@mf\endcsname{\def\PY@tc##1{\textcolor[rgb]{0.40,0.40,0.40}{##1}}}
\expandafter\def\csname PY@tok@mh\endcsname{\def\PY@tc##1{\textcolor[rgb]{0.40,0.40,0.40}{##1}}}
\expandafter\def\csname PY@tok@mi\endcsname{\def\PY@tc##1{\textcolor[rgb]{0.40,0.40,0.40}{##1}}}
\expandafter\def\csname PY@tok@il\endcsname{\def\PY@tc##1{\textcolor[rgb]{0.40,0.40,0.40}{##1}}}
\expandafter\def\csname PY@tok@mo\endcsname{\def\PY@tc##1{\textcolor[rgb]{0.40,0.40,0.40}{##1}}}
\expandafter\def\csname PY@tok@ch\endcsname{\let\PY@it=\textit\def\PY@tc##1{\textcolor[rgb]{0.25,0.50,0.50}{##1}}}
\expandafter\def\csname PY@tok@cm\endcsname{\let\PY@it=\textit\def\PY@tc##1{\textcolor[rgb]{0.25,0.50,0.50}{##1}}}
\expandafter\def\csname PY@tok@cpf\endcsname{\let\PY@it=\textit\def\PY@tc##1{\textcolor[rgb]{0.25,0.50,0.50}{##1}}}
\expandafter\def\csname PY@tok@c1\endcsname{\let\PY@it=\textit\def\PY@tc##1{\textcolor[rgb]{0.25,0.50,0.50}{##1}}}
\expandafter\def\csname PY@tok@cs\endcsname{\let\PY@it=\textit\def\PY@tc##1{\textcolor[rgb]{0.25,0.50,0.50}{##1}}}

\def\PYZbs{\char`\\}
\def\PYZus{\char`\_}
\def\PYZob{\char`\{}
\def\PYZcb{\char`\}}
\def\PYZca{\char`\^}
\def\PYZam{\char`\&}
\def\PYZlt{\char`\<}
\def\PYZgt{\char`\>}
\def\PYZsh{\char`\#}
\def\PYZpc{\char`\%}
\def\PYZdl{\char`\$}
\def\PYZhy{\char`\-}
\def\PYZsq{\char`\'}
\def\PYZdq{\char`\"}
\def\PYZti{\char`\~}
% for compatibility with earlier versions
\def\PYZat{@}
\def\PYZlb{[}
\def\PYZrb{]}
\makeatother
  % ---
  % Pacotes adicionais, usados no anexo do modelo de folha de identificação
  % ---
  \usepackage{multicol}
  \usepackage{multirow}
  % ---

  % ---
  % Pacotes adicionais, usados apenas no âmbito do Modelo Canônico do abnteX2
  % ---
  \usepackage{lipsum}				% para geração de dummy text
  % ---

  % ---
  % Pacotes de citações
  % ---
  \usepackage[brazilian,hyperpageref]{backref}	 % Paginas com as citações na bibl
  \usepackage[alf]{abntex2cite}	% Citações padrão ABNT

  % ---
  % CONFIGURAÇÕES DE PACOTES
  % ---

  % ---
  % Configurações do pacote backref
  % Usado sem a opção hyperpageref de backref
  \renewcommand{\backrefpagesname}{Citado na(s) página(s):~}
  % Texto padrão antes do número das páginas
  \renewcommand{\backref}{}
  % Define os textos da citação
  \renewcommand*{\backrefalt}[4]{
  	\ifcase #1 %
  		Nenhuma citação no texto.%
  	\or
  		Citado na página #2.%
  	\else
  		Citado #1 vezes nas páginas #2.%
  	\fi}%
  % ---

  % ---
  % Informações de dados para CAPA e FOLHA DE ROSTO
  % ---
  \titulo{Projeto 4\\MAE001 - Modelagem Mat. em Finanças I \\CAPM \& Estrutura a Termo}
  \autor{Gil Sales M. Neto \& João Victor de Fonseca}
  \local{Brasil}
  \data{Junho, 2019}
  \instituicao{%
    Universidade Federal do Rio de Janeiro
    \par
    Instituto de Matemática
    \par
    Bacharelado em Matemática Aplicada
		\par
		Prof.: Marco Cabral}
  \tipotrabalho{Relatório técnico}
  % O preambulo deve conter o tipo do trabalho, o objetivo,
  % o nome da instituição e a área de concentração
  \preambulo{}
  % ---

  % ---
  % Configurações de aparência do PDF final

  % alterando o aspecto da cor azul
  \definecolor{blue}{RGB}{41,5,195}

  % informações do PDF
  \makeatletter
  \hypersetup{
       	%pagebackref=true,
  		pdftitle={\@title},
  		pdfauthor={\@author},
      	pdfsubject={\imprimirpreambulo},
  	    pdfcreator={LaTeX with abnTeX2},
  		pdfkeywords={abnt}{latex}{abntex}{abntex2}{relatório técnico},
  		colorlinks=true,       		% false: boxed links; true: colored links
      	linkcolor=blue,          	% color of internal links
      	citecolor=blue,        		% color of links to bibliography
      	filecolor=magenta,      		% color of file links
  		urlcolor=blue,
  		bookmarksdepth=4
  }
  \makeatother
  % ---

  % ---
  % Espaçamentos entre linhas e parágrafos
  % ---

  % O tamanho do parágrafo é dado por:
  \setlength{\parindent}{1.3cm}

  % Controle do espaçamento entre um parágrafo e outro:
  \setlength{\parskip}{0.2cm}  % tente também \onelineskip

  % ---
  % compila o indice
  % ---
  \makeindex
  % ---

  % ----
  % Início do documento
  % ----
\begin{document}

% Seleciona o idioma do documento (conforme pacotes do babel)
%\selectlanguage{english}
\selectlanguage{brazil}

% Retira espaço extra obsoleto entre as frases.
\frenchspacing

% ----------------------------------------------------------
% ELEMENTOS PRÉ-TEXTUAIS
% ----------------------------------------------------------
% \pretextual

% ---
% Capa
% ---
% ---

% ---
% Folha de rosto
% (o * indica que haverá a ficha bibliográfica)
% ---
\imprimirfolhaderosto*
% ---
% ---
% inserir o sumario
% ---

\tableofcontents*
\cleardoublepage

\chapter{Os Algoritmos}
Esta seção tem como objetivo apresentar todos os códigos utilizados nas simulações.
\section{Funções principais}

Utilizamos a linguagem {\color{green}Python3} para a implementação do algoritmo que determina a estrutura a termo de juros.
Esse algoritmo tem depedência dos pacotes: {\color{red}Scipy} para Spline, {\color{red}Numpy} e {\color{red}MatPlotLib.pyplot} para os gráficos.

\begin{Verbatim}[breaklines=true, commandchars=\\\{\}]
{\color{incolor}In [{\color{incolor}165}]:} \PY{k+kn}{import} \PY{n+nn}{numpy} \PY{k}{as} \PY{n+nn}{np}
          \PY{k+kn}{import} \PY{n+nn}{matplotlib}\PY{n+nn}{.}\PY{n+nn}{pyplot} \PY{k}{as} \PY{n+nn}{plt}
          \PY{k+kn}{from} \PY{n+nn}{scipy}\PY{n+nn}{.}\PY{n+nn}{interpolate} \PY{k}{import} \PY{n}{spline}
\end{Verbatim}

\begin{Verbatim}[commandchars=\\\{\}]
{\color{incolor}In [{\color{incolor}259}]:} \PY{c+c1}{\PYZsh{} Carregando dados}
      \PY{n}{ltns\PYZus{}2017} \PY{o}{=} \PY{n}{np}\PY{o}{.}\PY{n}{array}\PY{p}{(}\PY{p}{[}\PY{p}{[}\PY{l+m+mf}{897.78}\PY{p}{,}\PY{l+m+mi}{2018}\PY{p}{]}\PY{p}{,}\PY{p}{[}\PY{l+m+mf}{812.14}\PY{p}{,}\PY{l+m+mi}{2019}\PY{p}{]}\PY{p}{,}\PY{p}{[}\PY{l+m+mf}{754.04}\PY{p}{,}\PY{l+m+mi}{2020}\PY{p}{]}\PY{p}{,}\PY{p}{[}\PY{l+m+mf}{651.41}\PY{p}{,}\PY{l+m+mi}{2021}\PY{p}{]}\PY{p}{]}\PY{p}{)}
      \PY{n}{ltns\PYZus{}2018} \PY{o}{=} \PY{n}{np}\PY{o}{.}\PY{n}{array}\PY{p}{(}\PY{p}{[}\PY{p}{[}\PY{l+m+mf}{936.03}\PY{p}{,}\PY{l+m+mi}{2019}\PY{p}{]}\PY{p}{,}\PY{p}{[}\PY{l+m+mf}{855.85}\PY{p}{,}\PY{l+m+mi}{2020}\PY{p}{]}\PY{p}{,}\PY{p}{[}\PY{l+m+mf}{770.38}\PY{p}{,}\PY{l+m+mi}{2021}\PY{p}{]}\PY{p}{,}\PY{p}{[}\PY{l+m+mf}{622.14}\PY{p}{,}\PY{l+m+mi}{2023}\PY{p}{]}\PY{p}{]}\PY{p}{)}
      \PY{n}{ltns\PYZus{}2019} \PY{o}{=} \PY{n}{np}\PY{o}{.}\PY{n}{array}\PY{p}{(}\PY{p}{[}\PY{p}{[}\PY{l+m+mf}{937.58}\PY{p}{,}\PY{l+m+mi}{2020}\PY{p}{]}\PY{p}{,}\PY{p}{[}\PY{l+m+mf}{866.29}\PY{p}{,}\PY{l+m+mi}{2021}\PY{p}{]}\PY{p}{,}\PY{p}{[}\PY{l+m+mf}{804.61}\PY{p}{,}\PY{l+m+mi}{2022}\PY{p}{]}\PY{p}{,}\PY{p}{[}\PY{l+m+mf}{719.12}\PY{p}{,}\PY{l+m+mi}{2023}\PY{p}{]}\PY{p}{]}\PY{p}{)}
\end{Verbatim}

\newpage
\subsection{Carregando dados e Calculando Estrutura a Termo}

\begin{Verbatim}[breaklines=true,commandchars=\\\{\}]
{\color{incolor}In [{\color{incolor}260}]:} \PY{c+c1}{\PYZsh{} Algoritmo para estrutura a termo}
          \PY{k}{def} \PY{n+nf}{term\PYZus{}structure}\PY{p}{(}\PY{n}{valores}\PY{p}{,} \PY{n}{ano}\PY{p}{)}\PY{p}{:}
              \PY{n}{ys} \PY{o}{=} \PY{n}{np}\PY{o}{.}\PY{n}{array}\PY{p}{(}\PY{p}{[}\PY{p}{(}\PY{l+m+mi}{1000}\PY{o}{/}\PY{n}{i}\PY{p}{[}\PY{l+m+mi}{0}\PY{p}{]}\PY{p}{)}\PY{o}{*}\PY{o}{*}\PY{p}{(}\PY{l+m+mi}{1}\PY{o}{/}\PY{p}{(}\PY{n}{i}\PY{p}{[}\PY{l+m+mi}{1}\PY{p}{]}\PY{o}{\PYZhy{}}\PY{n}{ano}\PY{p}{)}\PY{p}{)} \PY{o}{\PYZhy{}} \PY{l+m+mi}{1} \PY{k}{for} \PY{n}{i} \PY{o+ow}{in} \PY{n}{valores}\PY{p}{]}\PY{p}{)}
              \PY{n}{ys} \PY{o}{*}\PY{o}{=} \PY{l+m+mi}{100}
              \PY{k}{return} \PY{n}{ys}

          \PY{c+c1}{\PYZsh{} Estrutura a termo para cada ano}
          \PY{n}{ts\PYZus{}2017} \PY{o}{=} \PY{p}{[}\PY{p}{(}\PY{n}{i}\PY{p}{[}\PY{l+m+mi}{1}\PY{p}{]}\PY{o}{\PYZhy{}}\PY{l+m+mi}{2017}\PY{p}{)} \PY{k}{for} \PY{n}{i} \PY{o+ow}{in} \PY{n}{ltns\PYZus{}2017}\PY{p}{]}
          \PY{n}{ys\PYZus{}2017} \PY{o}{=} \PY{n}{term\PYZus{}structure}\PY{p}{(}\PY{n}{ltns\PYZus{}2017}\PY{p}{,} \PY{l+m+mi}{2017}\PY{p}{)}

          \PY{n}{ts\PYZus{}2018} \PY{o}{=} \PY{p}{[}\PY{p}{(}\PY{n}{i}\PY{p}{[}\PY{l+m+mi}{1}\PY{p}{]}\PY{o}{\PYZhy{}}\PY{l+m+mi}{2018}\PY{p}{)} \PY{k}{for} \PY{n}{i} \PY{o+ow}{in} \PY{n}{ltns\PYZus{}2018}\PY{p}{]}
          \PY{n}{ys\PYZus{}2018} \PY{o}{=} \PY{n}{term\PYZus{}structure}\PY{p}{(}\PY{n}{ltns\PYZus{}2018}\PY{p}{,} \PY{l+m+mi}{2018}\PY{p}{)}

          \PY{n}{ts\PYZus{}2019} \PY{o}{=} \PY{p}{[}\PY{p}{(}\PY{n}{i}\PY{p}{[}\PY{l+m+mi}{1}\PY{p}{]}\PY{o}{\PYZhy{}}\PY{l+m+mi}{2019}\PY{p}{)} \PY{k}{for} \PY{n}{i} \PY{o+ow}{in} \PY{n}{ltns\PYZus{}2019}\PY{p}{]}
          \PY{n}{ys\PYZus{}2019} \PY{o}{=} \PY{n}{term\PYZus{}structure}\PY{p}{(}\PY{n}{ltns\PYZus{}2019}\PY{p}{,} \PY{l+m+mi}{2019}\PY{p}{)}
\end{Verbatim}
\newpage

\subsection{Plots dos gráficos}

\subsubsection{Função para plotagem}
\begin{Verbatim}[commandchars=\\\{\}]
{\color{incolor}In [{\color{incolor}269}]:} \PY{c+c1}{\PYZsh{} Funções para plotar gráficos}
      \PY{k}{def} \PY{n+nf}{plot\PYZus{}spline}\PY{p}{(}\PY{n}{ts}\PY{p}{,} \PY{n}{ys}\PY{p}{,} \PY{n}{ano}\PY{p}{,} \PY{n}{lim}\PY{p}{)}\PY{p}{:}
          \PY{n}{xs} \PY{o}{=} \PY{n}{np}\PY{o}{.}\PY{n}{linspace}\PY{p}{(}\PY{l+m+mi}{1}\PY{p}{,}\PY{n}{lim}\PY{p}{,}\PY{l+m+mi}{100}\PY{p}{)}
          \PY{n}{spl} \PY{o}{=} \PY{n}{spline}\PY{p}{(}\PY{n}{ts}\PY{p}{,} \PY{n}{ys}\PY{p}{,} \PY{n}{xs}\PY{p}{)}
          \PY{n}{plt}\PY{o}{.}\PY{n}{figure}\PY{p}{(}\PY{n}{figsize}\PY{o}{=}\PY{p}{(}\PY{l+m+mi}{12}\PY{p}{,}\PY{l+m+mi}{8}\PY{p}{)}\PY{p}{)}
          \PY{n}{plt}\PY{o}{.}\PY{n}{title}\PY{p}{(}\PY{l+s+s1}{\PYZsq{}}\PY{l+s+s1}{Yield Curve \PYZhy{} Titulos emitidos em }\PY{l+s+si}{\PYZpc{}d}\PY{l+s+s1}{\PYZsq{}} \PY{o}{\PYZpc{}} \PY{n}{ano}\PY{p}{)}
          \PY{n}{plt}\PY{o}{.}\PY{n}{xlabel}\PY{p}{(}\PY{l+s+s1}{\PYZsq{}}\PY{l+s+s1}{Tempo até maturidade (em anos)}\PY{l+s+s1}{\PYZsq{}}\PY{p}{)}
          \PY{n}{plt}\PY{o}{.}\PY{n}{ylabel}\PY{p}{(}\PY{l+s+s1}{\PYZsq{}}\PY{l+s+s1}{Juros (}\PY{l+s+s1}{\PYZpc{}}\PY{l+s+s1}{)}\PY{l+s+s1}{\PYZsq{}}\PY{p}{)}
          \PY{n}{plt}\PY{o}{.}\PY{n}{grid}\PY{p}{(}\PY{k+kc}{True}\PY{p}{)}
          \PY{n}{plt}\PY{o}{.}\PY{n}{plot}\PY{p}{(}\PY{n}{xs}\PY{p}{,}\PY{n}{spl}\PY{p}{,} \PY{l+s+s1}{\PYZsq{}}\PY{l+s+s1}{r}\PY{l+s+s1}{\PYZsq{}}\PY{p}{)}
          \PY{n}{plt}\PY{o}{.}\PY{n}{plot}\PY{p}{(}\PY{n}{ts}\PY{p}{,}\PY{n}{ys}\PY{p}{,} \PY{l+s+s1}{\PYZsq{}}\PY{l+s+s1}{ro}\PY{l+s+s1}{\PYZsq{}}\PY{p}{)}
          \PY{n}{plt}\PY{o}{.}\PY{n}{show}\PY{p}{(}\PY{p}{)}
\end{Verbatim}

\subsubsection{Plot Emitidos em 2017}
\begin{Verbatim}[commandchars=\\\{\}]
{\color{incolor}In [{\color{incolor}273}]:} \PY{n}{plot\PYZus{}spline}\PY{p}{(}\PY{n}{ts\PYZus{}2017}\PY{p}{,} \PY{n}{ys\PYZus{}2017}\PY{p}{,} \PY{l+m+mi}{2017}\PY{p}{,} \PY{l+m+mi}{4}\PY{p}{)}
\end{Verbatim}

\subsubsection{Plot Emitidos em 2018}
\begin{Verbatim}[commandchars=\\\{\}]
{\color{incolor}In [{\color{incolor}272}]:} \PY{n}{plot\PYZus{}spline}\PY{p}{(}\PY{n}{ts\PYZus{}2018}\PY{p}{,} \PY{n}{ys\PYZus{}2018}\PY{p}{,} \PY{l+m+mi}{2018}\PY{p}{,} \PY{l+m+mi}{5}\PY{p}{)}
\end{Verbatim}

\subsubsection{Plot Emitidos em 2019}
\begin{Verbatim}[commandchars=\\\{\}]
{\color{incolor}In [{\color{incolor}270}]:} \PY{n}{plot\PYZus{}spline}\PY{p}{(}\PY{n}{ts\PYZus{}2019}\PY{p}{,} \PY{n}{ys\PYZus{}2019}\PY{p}{,} \PY{l+m+mi}{2019}\PY{p}{,}\PY{l+m+mi}{4}\PY{p}{)}
\end{Verbatim}

\subsection{Atividade 3}
\subsubsection{Importando Dados}
\begin{Verbatim}[commandchars=\\\{\}]
{\color{incolor}In [{\color{incolor}90}]:} \PY{n}{kroton} \PY{o}{=} \PY{n}{pd}\PY{o}{.}\PY{n}{read\PYZus{}csv}\PY{p}{(}\PY{l+s+s1}{\PYZsq{}}\PY{l+s+s1}{KROT3.csv}\PY{l+s+s1}{\PYZsq{}}\PY{p}{)}\PY{p}{[}\PY{p}{:}\PY{p}{:}\PY{o}{\PYZhy{}}\PY{l+m+mi}{1}\PY{p}{]}
         \PY{n}{mglu} \PY{o}{=} \PY{n}{pd}\PY{o}{.}\PY{n}{read\PYZus{}csv}\PY{p}{(}\PY{l+s+s1}{\PYZsq{}}\PY{l+s+s1}{MGLU3.csv}\PY{l+s+s1}{\PYZsq{}}\PY{p}{)}\PY{p}{[}\PY{p}{:}\PY{p}{:}\PY{o}{\PYZhy{}}\PY{l+m+mi}{1}\PY{p}{]}
         \PY{n}{itau} \PY{o}{=} \PY{n}{pd}\PY{o}{.}\PY{n}{read\PYZus{}csv}\PY{p}{(}\PY{l+s+s1}{\PYZsq{}}\PY{l+s+s1}{ITUB4.csv}\PY{l+s+s1}{\PYZsq{}}\PY{p}{)}\PY{p}{[}\PY{p}{:}\PY{p}{:}\PY{o}{\PYZhy{}}\PY{l+m+mi}{1}\PY{p}{]}
         \PY{n}{raias} \PY{o}{=} \PY{n}{pd}\PY{o}{.}\PY{n}{read\PYZus{}csv}\PY{p}{(}\PY{l+s+s1}{\PYZsq{}}\PY{l+s+s1}{RADL3.csv}\PY{l+s+s1}{\PYZsq{}}\PY{p}{)}\PY{p}{[}\PY{p}{:}\PY{p}{:}\PY{o}{\PYZhy{}}\PY{l+m+mi}{1}\PY{p}{]}
\end{Verbatim}
\subsubsection{Preparando Dados}
\begin{Verbatim}[commandchars=\\\{\}]
{\color{incolor}In [{\color{incolor}91}]:} \PY{n}{itau\PYZus{}fecho} \PY{o}{=} \PY{n}{itau}\PY{o}{.}\PY{n}{iloc}\PY{p}{[}\PY{l+m+mi}{0}\PY{p}{:}\PY{p}{,}\PY{l+m+mi}{1}\PY{p}{:}\PY{l+m+mi}{2}\PY{p}{]}\PY{o}{.}\PY{n}{values}\PY{o}{.}\PY{n}{flatten}\PY{p}{(}\PY{p}{)}
		 \PY{n}{kroton\PYZus{}fecho} \PY{o}{=} \PY{n}{kroton}\PY{o}{.}\PY{n}{iloc}\PY{p}{[}\PY{l+m+mi}{0}\PY{p}{:}\PY{p}{,}\PY{l+m+mi}{1}\PY{p}{:}\PY{l+m+mi}{2}\PY{p}{]}\PY{o}{.}\PY{n}{values}\PY{o}{.}\PY{n}{flatten}\PY{p}{(}\PY{p}{)}
		 \PY{n}{raias\PYZus{}fecho} \PY{o}{=} \PY{n}{raias}\PY{o}{.}\PY{n}{iloc}\PY{p}{[}\PY{l+m+mi}{0}\PY{p}{:}\PY{p}{,}\PY{l+m+mi}{1}\PY{p}{:}\PY{l+m+mi}{2}\PY{p}{]}\PY{o}{.}\PY{n}{values}\PY{o}{.}\PY{n}{flatten}\PY{p}{(}\PY{p}{)}
		 \PY{n}{mglu\PYZus{}fecho} \PY{o}{=} \PY{n}{mglu}\PY{o}{.}\PY{n}{iloc}\PY{p}{[}\PY{l+m+mi}{0}\PY{p}{:}\PY{p}{,}\PY{l+m+mi}{1}\PY{p}{:}\PY{l+m+mi}{2}\PY{p}{]}\PY{o}{.}\PY{n}{values}\PY{o}{.}\PY{n}{flatten}\PY{p}{(}\PY{p}{)}

		 \PY{n}{beta\PYZus{}itau} \PY{o}{=} \PY{l+m+mf}{1.2663}
		 \PY{n}{rm\PYZus{}itau} \PY{o}{=} \PY{l+m+mf}{0.2039}

		 \PY{n}{beta\PYZus{}mglu} \PY{o}{=} \PY{l+m+mf}{1.2737}
		 \PY{n}{rm\PYZus{}mglu} \PY{o}{=} \PY{l+m+mf}{0.2598}

		 \PY{n}{beta\PYZus{}raia} \PY{o}{=} \PY{l+m+mf}{0.3499}
		 \PY{n}{rm\PYZus{}raia} \PY{o}{=} \PY{l+m+mf}{0.1397}

		 \PY{n}{beta\PYZus{}kroton} \PY{o}{=} \PY{l+m+mf}{1.1226}
		 \PY{n}{rm\PYZus{}kroton} \PY{o}{=} \PY{l+m+mf}{0.0738}

		 \PY{n}{re\PYZus{}itau} \PY{o}{=} \PY{n}{retorno\PYZus{}esp}\PY{p}{(}\PY{n}{selic}\PY{p}{,} \PY{n}{beta} \PY{o}{=} \PY{n}{beta\PYZus{}itau}\PY{p}{,} \PY{n}{rm} \PY{o}{=} \PY{n}{rm\PYZus{}itau}\PY{p}{)}
		 \PY{n}{re\PYZus{}mglu} \PY{o}{=} \PY{n}{retorno\PYZus{}esp}\PY{p}{(}\PY{n}{selic}\PY{p}{,} \PY{n}{beta} \PY{o}{=} \PY{n}{beta\PYZus{}mglu}\PY{p}{,} \PY{n}{rm} \PY{o}{=} \PY{n}{rm\PYZus{}mglu}\PY{p}{)}
		 \PY{n}{re\PYZus{}raia} \PY{o}{=} \PY{n}{retorno\PYZus{}esp}\PY{p}{(}\PY{n}{selic}\PY{p}{,} \PY{n}{beta} \PY{o}{=} \PY{n}{beta\PYZus{}raia}\PY{p}{,} \PY{n}{rm} \PY{o}{=} \PY{n}{rm\PYZus{}raia}\PY{p}{)}
		 \PY{n}{re\PYZus{}kroton} \PY{o}{=} \PY{n}{retorno\PYZus{}esp}\PY{p}{(}\PY{n}{selic}\PY{p}{,} \PY{n}{beta} \PY{o}{=} \PY{n}{beta\PYZus{}kroton}\PY{p}{,} \PY{n}{rm} \PY{o}{=} \PY{n}{rm\PYZus{}kroton}\PY{p}{)}

		 \PY{n}{vol\PYZus{}itau} \PY{o}{=} \PY{l+m+mf}{0.513346}
		 \PY{n}{vol\PYZus{}mglu} \PY{o}{=} \PY{l+m+mf}{0.726408}
		 \PY{n}{vol\PYZus{}raia} \PY{o}{=} \PY{l+m+mf}{0.164342}
		 \PY{n}{vol\PYZus{}kroton} \PY{o}{=} \PY{l+m+mf}{0.226435}

		 \PY{n}{shp\PYZus{}itau} \PY{o}{=} \PY{n}{sharpe}\PY{p}{(}\PY{n}{vol\PYZus{}itau}\PY{p}{,} \PY{n}{re\PYZus{}itau}\PY{p}{)}
		 \PY{n}{shp\PYZus{}mglu} \PY{o}{=} \PY{n}{sharpe}\PY{p}{(}\PY{n}{vol\PYZus{}mglu}\PY{p}{,} \PY{n}{re\PYZus{}mglu}\PY{p}{)}
		 \PY{n}{shp\PYZus{}raia} \PY{o}{=} \PY{n}{sharpe}\PY{p}{(}\PY{n}{vol\PYZus{}raia}\PY{p}{,} \PY{n}{re\PYZus{}raia}\PY{p}{)}
		 \PY{n}{shp\PYZus{}kroton} \PY{o}{=} \PY{n}{sharpe}\PY{p}{(}\PY{n}{vol\PYZus{}kroton}\PY{p}{,} \PY{n}{re\PYZus{}kroton}\PY{p}{)}
\end{Verbatim}

\subsubsection{Algoritmos}
\begin{Verbatim}[commandchars=\\\{\}]
{\color{incolor}In [{\color{incolor}92}]:} \PY{n}{selic} \PY{o}{=} \PY{l+m+mf}{0.065}
		 \PY{k}{def} \PY{n+nf}{retorno\PYZus{}esp}\PY{p}{(}\PY{n}{rf} \PY{o}{=} \PY{n}{selic}\PY{p}{,} \PY{n}{beta} \PY{o}{=} \PY{l+m+mi}{1}\PY{p}{,} \PY{n}{rm} \PY{o}{=} \PY{l+m+mi}{1}\PY{p}{)}\PY{p}{:}
				 \PY{k}{return} \PY{n}{rf} \PY{o}{+} \PY{n}{beta}\PY{o}{*}\PY{p}{(}\PY{n}{rm} \PY{o}{\PYZhy{}} \PY{n}{rf}\PY{p}{)}

		 \PY{k}{def} \PY{n+nf}{sharpe}\PY{p}{(}\PY{n}{vol}\PY{p}{,} \PY{n}{re}\PY{p}{,}\PY{n}{rf} \PY{o}{=} \PY{n}{selic}\PY{p}{)}\PY{p}{:}
				 \PY{k}{return} \PY{p}{(}\PY{n}{re} \PY{o}{\PYZhy{}} \PY{n}{rf}\PY{p}{)}\PY{o}{/}\PY{n}{vol}
\end{Verbatim}

\subsubsection{Plot Retorno}
\begin{Verbatim}[commandchars=\\\{\}]
{\color{incolor}In [{\color{incolor}93}]:} \PY{n}{res} \PY{o}{=} \PY{n}{np}\PY{o}{.}\PY{n}{array}\PY{p}{(}\PY{p}{[}\PY{n}{re\PYZus{}itau}\PY{p}{,} \PY{n}{re\PYZus{}mglu}\PY{p}{,} \PY{n}{re\PYZus{}raia}\PY{p}{,} \PY{n}{re\PYZus{}kroton}\PY{p}{]}\PY{p}{)}
		 \PY{n}{plt}\PY{o}{.}\PY{n}{figure}\PY{p}{(}\PY{n}{figsize}\PY{o}{=}\PY{p}{(}\PY{l+m+mi}{12}\PY{p}{,}\PY{l+m+mi}{8}\PY{p}{)}\PY{p}{)}
		 \PY{n}{plt}\PY{o}{.}\PY{n}{bar}\PY{p}{(}\PY{p}{[}\PY{l+s+s1}{\PYZsq{}}\PY{l+s+s1}{Itau}\PY{l+s+s1}{\PYZsq{}}\PY{p}{,} \PY{l+s+s1}{\PYZsq{}}\PY{l+s+s1}{Magazine Luiza}\PY{l+s+s1}{\PYZsq{}}\PY{p}{,} \PY{l+s+s1}{\PYZsq{}}\PY{l+s+s1}{Drogas Raia}\PY{l+s+s1}{\PYZsq{}}\PY{p}{,} \PY{l+s+s1}{\PYZsq{}}\PY{l+s+s1}{Kroton}\PY{l+s+s1}{\PYZsq{}}\PY{p}{]}\PY{p}{,} \PY{n}{res}\PY{o}{*}\PY{l+m+mi}{100}\PY{p}{)}
		 \PY{n}{plt}\PY{o}{.}\PY{n}{ylabel}\PY{p}{(}\PY{l+s+s1}{\PYZsq{}}\PY{l+s+s1}{Retorno (em }\PY{l+s+s1}{\PYZpc{}}\PY{l+s+s1}{)}\PY{l+s+s1}{\PYZsq{}}\PY{p}{)}
		 \PY{n}{plt}\PY{o}{.}\PY{n}{grid}\PY{p}{(}\PY{n}{alpha} \PY{o}{=} \PY{l+m+mf}{0.4}\PY{p}{)}
		 \PY{n}{plt}\PY{o}{.}\PY{n}{show}\PY{p}{(}\PY{p}{)}
\end{Verbatim}

\subsubsection{Plot Sharpe}
\begin{Verbatim}[commandchars=\\\{\}]
{\color{incolor}In [{\color{incolor}94}]:} \PY{n}{sharpes} \PY{o}{=} \PY{p}{[}\PY{n}{shp\PYZus{}itau}\PY{p}{,} \PY{n}{shp\PYZus{}mglu}\PY{p}{,} \PY{n}{shp\PYZus{}raia}\PY{p}{,} \PY{n}{shp\PYZus{}kroton}\PY{p}{]}
		 \PY{n}{plt}\PY{o}{.}\PY{n}{figure}\PY{p}{(}\PY{n}{figsize}\PY{o}{=}\PY{p}{(}\PY{l+m+mi}{12}\PY{p}{,}\PY{l+m+mi}{8}\PY{p}{)}\PY{p}{)}
		 \PY{n}{plt}\PY{o}{.}\PY{n}{bar}\PY{p}{(}\PY{p}{[}\PY{l+s+s1}{\PYZsq{}}\PY{l+s+s1}{Itau}\PY{l+s+s1}{\PYZsq{}}\PY{p}{,} \PY{l+s+s1}{\PYZsq{}}\PY{l+s+s1}{Magazine Luiza}\PY{l+s+s1}{\PYZsq{}}\PY{p}{,} \PY{l+s+s1}{\PYZsq{}}\PY{l+s+s1}{Drogas Raia}\PY{l+s+s1}{\PYZsq{}}\PY{p}{,} \PY{l+s+s1}{\PYZsq{}}\PY{l+s+s1}{Kroton}\PY{l+s+s1}{\PYZsq{}}\PY{p}{]}\PY{p}{,} \PY{n}{sharpes}\PY{p}{)}
		 \PY{n}{plt}\PY{o}{.}\PY{n}{grid}\PY{p}{(}\PY{n}{alpha} \PY{o}{=} \PY{l+m+mf}{0.4}\PY{p}{)}
		 \PY{n}{plt}\PY{o}{.}\PY{n}{ylabel}\PY{p}{(}\PY{l+s+s1}{\PYZsq{}}\PY{l+s+s1}{Índice Sharpe}\PY{l+s+s1}{\PYZsq{}}\PY{p}{)}
		 \PY{n}{plt}\PY{o}{.}\PY{n}{show}\PY{p}{(}\PY{p}{)}
\end{Verbatim}









\chapter{Atividade 1}
Usamos como fonte de dados para valor das LTNs o histórico fornecido pelo site do Tesouro Nacional: \url{https://sisweb.tesouro.gov.br/apex/f?p=2031:2:0::::}, os dados obtidos foram transcritos diretamente para listas (ltns\_2017, ltns\_2018, ltns\_2019)\\
Utilizamos os arquivos LTN dos anos 2017, 2018 e 2019\\
Todos os valores usados levam em conta emissão em 01 de Janeiro do ano corrente, e maturidade em 01 de Janeiro do ano de maturidade, ano este que varia para cada ano de emissão.\\

\subsection{Estrutura a termo de juros}
Os juros foram calculados utilizando o modelo para estrutura a termo fornecido no livro texto:
\[
y_m = \left(\frac{V_f}{B_{0,m}}\right)^\frac{1}{m} - 1
\]
Onde:\\
\(y_m\) é o rendimento/juros anual para o título. \\
\(m\) é o tempo restante para maturidade, em anos.\\
\(B_{0,m}\) é o valor em tempo 0 do título com maturidade em tempo \(m\)\\
\(V_f\) é o valor de face, que no caso das LTNs: \(V_f = 1000\)
\\
Com \(y_m\) calculados, plotamos a Curva de Rendimento (Yield Curve), ou estrutura a termo de juros.\\
O Gráfico é feito com tempo para maturidade no eixo horizontal e rendimento no eixo vertical, os pontos são interpolados com Spline do pacote SciPy.

\newpage
\subsection{Estrutura a Termo para Título emitido em 2017}
\begin{tabular}{ll}
\toprule
\multicolumn{1}{c}{Ano de Maturidade} & Valor em tempo 0 \\
\midrule
2018 & R\$ 897.78 \\
2019 & R\$ 812.14 \\
2020 & R\$ 754.04\\
2021 & R\$ 651.41
\end{tabular}
\begin{figure}[h]
	\adjustimage{max size={1\linewidth}{1\paperheight}}{y_2017.png}
\end{figure}

\newpage
\subsection{Estrutura a Termo para Título emitido em 2018}
\begin{tabular}{ll}
\toprule
\multicolumn{1}{c}{Ano de Maturidade} & Valor em tempo 0 \\
\midrule
2019 & R\$ 936.03 \\
2020 & R\$ 855.85 \\
2021 & R\$ 770.38\\
2023 & R\$ 622.14
\end{tabular}
\begin{figure}[h]
	\adjustimage{max size={1\linewidth}{1\paperheight}}{y_2018.png}
\end{figure}

\newpage
\subsection{Estrutura a Termo para Título emitido em 2019}
\begin{tabular}{ll}
\toprule
\multicolumn{1}{c}{Ano de Maturidade} & Valor em tempo 0 \\
\midrule
2018 & R\$ 897.78 \\
2019 & R\$ 812.14 \\
2020 & R\$ 754.04\\
2021 & R\$ 651.41
\end{tabular}

\begin{figure}[h]
	\adjustimage{max size={1\linewidth}{1\paperheight}}{y_2019.png}
\end{figure}

\chapter{Atividade 3}
Usamos como fonte de dados para valor das ações o histórico fornecido pelo site Investing.com: \url{https://br.investing.com/equities/}, os dados foram salvos em um arquivo .csv e importado com a biblioteca Pandas.\\
As empresas e ações utilizadas foram: \\
\begin{itemize}
	\item Itaú Unibanco - ITUB4
	\item Kroton Educacional - KROT3
	\item Magazine Luiza - MGLU3
	\item Drogas Raia - RADL3
\end{itemize}

\subsection{Retorno esperado}
\begin{figure}[h]
	\adjustimage{max size={1\linewidth}{1\paperheight}}{retorno.png}
\end{figure}

\subsection{Índice Sharpe}
\begin{figure}[h]
	\adjustimage{max size={1\linewidth}{1\paperheight}}{sharpe.png}
\end{figure}

\end{document}
