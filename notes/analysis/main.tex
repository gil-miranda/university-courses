\documentclass{article}
%\documentclass[runningheads]{llncs}
%
%\usepackage{graphicx}
% Used for displaying a sample figure. If possible, figure files should
% be included in EPS format.
%
% If you use the hyperref package, please uncomment the following line
% to display URLs in blue roman font according to Springer's eBook style:
% \renewcommand\UrlFont{\color{blue}\rmfamily}
%\documentclass[12pt]{amsart}
\RequirePackage{etex}

\usepackage[margin=1in]{geometry}
\usepackage[english]{babel}
\usepackage[utf8]{inputenc}
\usepackage{subcaption}
\usepackage{amsmath}
\usepackage{amssymb}
\usepackage{amsfonts}
\usepackage{amsthm}
\usepackage{mathrsfs}
\usepackage[all]{xy}
\usepackage[pdftex]{graphicx}
\usepackage{color}
\usepackage{cite}
\usepackage{url}
\usepackage{indent first}
\usepackage[labelfont=bf,labelsep=period,justification=raggedright]{caption}
\usepackage[english]{babel}
\usepackage[utf8]{inputenc}
\usepackage[colorlinks=true,linkcolor=blue]{hyperref}
\usepackage[colorinlistoftodos]{todonotes}
\usepackage{tkz-fct}
\usepackage{tikz}
\usetikzlibrary{calc}
\usepackage{multicol}
\PassOptionsToPackage{dvipsnames,svgnames}{xcolor}
\usepackage{textcomp}


% \topmargin 0.4cm
% \oddsidemargin 0.5cm
% \evensidemargin 0.5cm
% \textwidth 14cm 
% \textheight 20.2cm

% \setlength{\oddsidemargin}{0.25in}
% \setlength{\evensidemargin}{0.25in}
% \setlength{\textwidth}{6in}
%\setlength{\topmargin}{-0.25in}
%\setlength{\textheight}{8in}

\DeclareMathOperator{\ab}{ab}
%\DeclareMathOperator{\arg}{arg}
\DeclareMathOperator{\Aut}{Aut}
\DeclareMathOperator{\BGL}{BGL}
\DeclareMathOperator{\Br}{Br}
\DeclareMathOperator{\card}{card}
\DeclareMathOperator{\ch}{ch}
\DeclareMathOperator{\Char}{char}
\DeclareMathOperator{\CHur}{CHur}
\DeclareMathOperator{\Cl}{Cl}
\DeclareMathOperator{\coker}{coker}
\DeclareMathOperator{\Conf}{Conf}
\DeclareMathOperator{\disc}{disc}
\DeclareMathOperator{\End}{End}
\DeclareMathOperator{\et}{\text{\'et}}
\DeclareMathOperator{\Fix}{Fix}
\DeclareMathOperator{\Gal}{Gal}
\DeclareMathOperator{\GL}{GL}
\DeclareMathOperator{\Hom}{Hom}
\DeclareMathOperator{\Hur}{Hur}
\DeclareMathOperator{\im}{im}
\DeclareMathOperator{\Ind}{Ind}
\DeclareMathOperator{\Inn}{Inn}
\DeclareMathOperator{\Irr}{Irr}
\DeclareMathOperator{\lcm}{lcm}
\DeclareMathOperator{\Mor}{Mor}
\DeclareMathOperator{\ord}{ord}
\DeclareMathOperator{\Out}{Out}
\DeclareMathOperator{\Perm}{Perm}
\DeclareMathOperator{\PGL}{PGL}
\DeclareMathOperator{\Pin}{Pin}
\DeclareMathOperator{\PSL}{PSL}
\DeclareMathOperator{\rad}{rad}
%\DeclareMathOperator{\Re}{Re}
\DeclareMathOperator{\SL}{SL}
\DeclareMathOperator{\SO}{SO}
\DeclareMathOperator{\Spec}{Spec}
\DeclareMathOperator{\Spin}{Spin}
\DeclareMathOperator{\St}{St}
\DeclareMathOperator{\Surj}{Surj}
\DeclareMathOperator{\Syl}{Syl}
\DeclareMathOperator{\tame}{tame}
\DeclareMathOperator{\Tr}{Tr}
\DeclareMathOperator{\fancyC}{Č}

\newcommand{\eps}{\varepsilon}
\newcommand{\QED}{\hspace{\stretch{1}} $\blacksquare$}
\renewcommand{\AA}{\mathbb{A}}
\newcommand{\CC}{\mathbb{C}}
\newcommand{\EE}{\mathbb{E}}
\newcommand{\FF}{\mathbb{F}}
\newcommand{\HH}{\mathbb{H}}
\newcommand{\NN}{\mathbb{N}}
\newcommand{\OO}{\mathbb{O}}
\newcommand{\PP}{\mathbb{P}}
\newcommand{\QQ}{\mathbb{Q}}
\newcommand{\RR}{\mathbb{R}}
\newcommand{\ZZ}{\mathbb{Z}}
\newcommand{\bfm}{\mathbf{m}}
\newcommand{\mcA}{\mathcal{A}}
\newcommand{\mcG}{\mathcal{G}}
\newcommand{\mcH}{\mathcal{H}}
\newcommand{\mcM}{\mathcal{M}}
\newcommand{\mcN}{\mathcal{N}}
\newcommand{\mcO}{\mathcal{O}}
\newcommand{\mcP}{\mathcal{P}}
\newcommand{\mcQ}{\mathcal{Q}}
\newcommand{\mfa}{\mathfrak{a}}
\newcommand{\mfb}{\mathfrak{b}}
\newcommand{\mfc}{\mathfrak{c}}
\newcommand{\mfI}{\mathfrak{I}}
\newcommand{\mfM}{\mathfrak{M}}
\newcommand{\mfm}{\mathfrak{m}}
\newcommand{\mfo}{\mathfrak{o}}
\newcommand{\mfO}{\mathfrak{O}}
\newcommand{\mfP}{\mathfrak{P}}
\newcommand{\mfp}{\mathfrak{p}}
\newcommand{\mfq}{\mathfrak{q}}
\newcommand{\mfz}{\mathfrak{z}}
\newcommand{\msP}{\mathscr{P}}
\newcommand{\AGL}{\mathbb{A}\GL}
\newcommand{\Qbar}{\overline{\QQ}}
\renewcommand{\qedsymbol}{$\blacksquare$}

\DeclareRobustCommand{\sstirling}{\genfrac\{\}{0pt}{}}
\DeclareRobustCommand{\fstirling}{\genfrac[]{0pt}{}}
\def\multiset#1#2{\ensuremath{\left(\kern-.3em\left(\genfrac{}{}{0pt}{}{#1}{#2}\right)\kern-.3em\right)}}

\newcommand{\planefig}[2] {
\filldraw[shift={#1},rotate=#2] (.4,.3) circle (2pt);
\filldraw[shift={#1},rotate=#2] (-.4,.3) circle (2pt);
\draw[shift={#1},rotate=#2] (-20:.55) arc (-20:-160:.55);
\draw[shift={#1},rotate=#2] (0,0) circle (1cm);
}

\newcommand{\simon}[1]{\todo[color=green]{SR: #1}}
\newcommand{\nitya}[1]{\todo[color=blue!30]{NM: #1}}
\newcommand{\penghui}[1]{\todo[color=red!60]{HPH: #1}}
\newcommand{\michael}[1]{\todo[color=yellow]{MW: #1}}
\newcommand{\winnie}[1]{\todo[color=purple!60]{WL: #1}}
\newcommand{\peter}[1]{\todo[color=pink]{PR: #1}}
\newcommand{\jae}[1]{\todo[color=brown!60]{JL: #1}}
\newcommand{\silas}[1]{\todo[color=orange]{SJ: #1}}
\newcommand{\refr}[1]{\textcolor{blue}{#1}}


\theoremstyle{plain}
\newtheorem{thm}{Teorema}
%\newtheorem{lemma}[thm]{Lemma}
\newtheorem{cor}{Corolário}
\newtheorem{conj}{Conjectura}
\newtheorem{prop}{Proposição}
\newtheorem{lemma}{Lema}
\newtheorem{heur}{Heuristica}
\newtheorem{qn}{Questão}
%\newtheorem{claim}[thm]{Claim}
\newtheorem{axm}{Axioma}
\newtheorem{defn}{Definição}
\newtheorem{cond}{Condições}
\newtheorem*{notn}{Notação}

\theoremstyle{remark}
\newtheorem{rem}{Remark}
\newtheorem*{ex}{Exemplo}
\newtheorem*{exer}{Exercicio}

\numberwithin{equation}{section}
\numberwithin{thm}{section}
\numberwithin{defn}{section}
\numberwithin{lemma}{section}
\numberwithin{axm}{section}

\usepackage{arxiv}

\usepackage[utf8]{inputenc} % allow utf-8 input
\usepackage[T1]{fontenc}    % use 8-bit T1 fonts
\usepackage{hyperref}       % hyperlinks
\usepackage{url}            % simple URL typesetting
\usepackage{booktabs}       % professional-quality tables
\usepackage{amsfonts}       % blackboard math symbols
%\usepackage{nicefrac}       % compact symbols for 1/2, etc.
\usepackage{microtype}      % microtypography
%\usepackage{lipsum}

\title{Notas de estudo em Análise I (Análise Real)\\ Um guia de teoremas, resultados importantes\\ e exercícios}


\author{
  Gil S. M. Neto\\
  Graduando em Matemática Aplicada - UFRJ\\
  \texttt{gilsmneto@gmail.com, gil.neto@ufrj.br}\\
  \texttt{http://mirandagil.github.io}
  %% examples of more authors
  %% \AND
  %% Coauthor \\
  %% Affiliation \\
  %% Address \\
  %% \texttt{email} \\
  %% \And
  %% Coauthor \\
  %% Affiliation \\
  %% Address \\
  %% \texttt{email} \\
  %% \And
  %% Coauthor \\
  %% Affiliation \\
  %% Address \\
  %% \texttt{email} \\
}
\date{Última atualização: \today}

\begin{document}

\maketitle

\tableofcontents
\newpage

\section{Teoria Ingênua dos Conjuntos}

\begin{defn}[Informal de conjuntos]\phantomsection\label{def1:1}
Um conjunto é uma coleção não ordenada de objetos. Se \(x\) é um objeto do conjunto \(A\), dizemos \(x \in A\), caso contrário dizemos \(x \not\in A\). \\
Exemplo: \(3 \in \{1,2,3,4,5\}; \, \, 7 \not\in \{1,2,3,4,5\}\)
\end{defn}

	\begin{axm} [Conjuntos são objetos]\phantomsection\label{axm1:1}
		Se \(A\) é um conjunto, então \(A\) também é um objeto, ou seja, se existe outro conjunto \(B\), então faz sentido inferir \(A \in B\) ou \(A \not\in B\)
\end{axm}

\begin{ex}
Seja \(B = \{ 1, 3, \{4, 5\}, 8 \}; \,\, A = \{4,5\}\), então \(A \in B\)\\
Seja \(C = \{ 1, 3, 4, 5, 8 \}; \,\, D = \{4,5\}\), então \(C \subset D\)\\
é importante notar que apesar de \(4 \in A, 5 \in A\), é verdade que \(4 \not\in B, 5 \not\in B\) (verificar)
\end{ex}

\begin{defn}[Subconjuntos]\phantomsection\label{def1:2}
  \(A \subset B \iff x \in A \implies x \in B, \, \forall x \in A\)
\end{defn}

\begin{defn}[Igualdade de Conjuntos] \phantomsection\label{def1:3}
Definimos dois conjuntos \(A = B \iff A \subset B \wedge B \subset A\)\\
Ou seja, \(x \in A \implies x \in B, \, \forall x \in A \wedge y \in B \implies y \in A, \, \forall y \in B\)
\end{defn}

 \begin{axm}[Conjunto Vazio]\phantomsection\label{axm1:2}
 Existe um conjunto ao qual nenhum objeto pertence. A este grupo denominamos \(\emptyset\) .\\
 Para qualquer objeto \(x\), temos \(x \not\in \emptyset\).\\

\begin{lemma}[O Conjunto vazio é subconjunto de todo conjunto]\phantomsection\label{lem1:1}
 Seja \(A\) um conjunto qualquer, então \(\emptyset \subset A\)\\
 \begin{proof}
 Suponha que \(\emptyset \not\subset A\), para qualquer conjunto \(A\). Para negar a Definição \ref{def1:2} teremos: \(A \not\subset B \iff \exists \, x \in A; x \not\in B\) \\
 Logo, para termos \(\emptyset \not\subset A\), deve existir um objeto em \(\emptyset\) que não está contido em \(A\), mas não há nenhum objeto em \(\emptyset\), logo uma contradição, e temos \(\emptyset \subset A, \, \forall A\)
 \end{proof}
\end{lemma} 
 
 \begin{lemma}[O conjunto vazio é único]\phantomsection\label{lem1:2}
 	\begin{proof}
			 Seja \(\emptyset, \emptyset'\) conjuntos vazios, então do Lema \ref{lem1:1} temos \(\emptyset \subset 	\emptyset'\) e \(\emptyset' \subset \emptyset\), e pela Definição \ref{def1:3} \(\emptyset = \emptyset'\).
	 \end{proof}
 \end{lemma}
 \end{axm}
 
 \begin{lemma}[Escolha única]\phantomsection\label{lem1:3}
 Seja \(A\) um conjunto não vazio, então existe ao menos um \(x\) tal que \(x \in A\)\\
 \begin{proof}
 Suponha que não exista nenhum objeto x pertencente a \(A\), então: \(x \not\in A, \, \forall x\), mas isso implicaria que A é um conjunto vazio, o que contraria a hipótese.
 \end{proof}
\end{lemma}
Este lema nos permite escolher algum elemento de A. Ainda mais, dado uma família finita de Conjuntos \(A_1, A_2, \dots, A_n \), podemos escolher um elemento de cada conjunto \(x_1, x_2, \dots, x_n\). Para o caso infinito cairá no Axioma da Escolha, assunto a ser desenvolvido em outro momento.

\begin{axm}[Singleton]\phantomsection\label{axm1:3}
Dado um objeto \(a\), então existe um conjunto  de apenas um elemento \(\{a\}\). Ou seja, para todo objeto \(x, x \in \{a\} \iff y = a\). Ainda mais, para todo objeto \(a, b\) existe um conjunto \(\{a,b\}\) onde \(\forall y, y \in \{a,b\} \iff y = a \vee y = b\) 

\end{axm}

\section{A construção dos números}
\subsection{Números Naturais e Inteiros \(\NN, \ZZ\)}

In this section, we list some analytic statements regarding the convergence of Dirichlet series. We omit the proof of most theorems in this section; they generally reduce to extensive computation. Still, they make good exercises for the reader. 

\begin{prop} \label{2.1}
Let $$f(n) = \sum_{n \ge 1} \frac{a(n)}{n^s}$$ be a Dirichlet series and let $S(x) = \sum_{n \le x} a(n)$, and suppose there exist constants $a$ and $b$ such that $|S(x)| \le ax^b$ for all large $x$. Then, $f(s)$ converges uniformly for $s$ in $$D(b, \delta, \epsilon) = \{\Re(s) \ge b + \delta, \arg(s-b) \le \pi/2 - \epsilon\}$$ for all $\delta, \epsilon \ge 0$, and it converges to an analytic function on the half plane $\Re(s) > b$. (Note that $\Re(s)$ denotes the real part of $s$.)
\end{prop}

\begin{lemma}
The Riemann zeta function $\zeta(s)$ has a meromorphic continuation to the half plane $\Re(s) > 0$ with a simple pole at $s = 1$.
\end{lemma}

\begin{lemma}
For $s$ real and $s > 1$, $$\frac{1}{s-1} \le \zeta(s) \le 1 + \frac{1}{s-1}$$ Hence, $\zeta(s)$ has a simple pole at $s = 1$ and $$\zeta(s) = \frac{1}{s-1} + \text{function holomorphic near } 1$$
\end{lemma}

\begin{proof}
This is left as an exercise to the reader. (Hint: Look at the graph of $y = x^{-s}$ and relate $\zeta(s)$ to the area under the curve.)
\end{proof}

Armed with this fact, we can look at other interesting Dirichlet series. 

\begin{prop}
Let $f(n)$ be a Dirichlet series for which there exists constants $C$, $a$, and $b < 1$ such that $|S(n) - an| \le Cx^b$. Then, $f$ extends to a meromorphic function on $\Re(s) > b$ with a simple pole at $s = 1$ with residue $a$. 
\end{prop}

\begin{proof}
For the Dirichlet series $f(s) - a\zeta(s)$, $|S(n)| \le Cx^b$, so by Proposition \ref{2.1}, this series converges for $\Re(s) > b$. The result readily follows.  
\end{proof}

Before we move on, we encounter one last lemma that will prove to be useful soon.

\begin{lemma} \label{2.5}
Let $u_1, u_2, \cdots$ be a sequence of real numbers $\ge 2$ for which $$f(s) = \prod_{j = 1}^{\infty} \frac{1}{1 - u_{j}^{-s}}$$ is uniformly convergent on each region $D(1, \delta, \epsilon)$ (with $\delta, \epsilon > 0$). Then, $$\log f(s) \sim \sum \frac{1}{u_{j}^{s}}$$ as $s \to 1^{+}$ (i.e., from the right side of the plane). 
\end{lemma}

\begin{proof}
This is a simple exercise in manipulating sums. (Hint: use the Maclaurin series for $\log(1-x)$ and then break the double sum apart.)
\end{proof}


\subsection{Números Racionais \(\QQ\) }

Now, we introduce some basic character theory. In particular, knowing certain statements about characters - namely, the orthogonality relations - will aid us in our study of L-functions.

\begin{defn}
A one-dimensional representation of a group $G$, i.e. $\chi: G \longrightarrow \CC^{\times}$ is a character of $G$. Note that this map is a homomorphism.
\end{defn}

\begin{prop} \label{3.2}
For a character $\chi$ of $G$, we have that $\sum_{a \in G} \chi(a) = \begin{cases}
|G| & \text{ if } \chi = \chi_{0} \text{ (the trivial character)} \\
0 &  \text { otherwise } \\
\end{cases}$  

\end{prop}

\begin{proof}
The first part is obvious. If we have a nontrivial character $\chi$, then for some $g \in G$, $\chi(g) \neq 1$. Then, $$\chi(g)\sum_{a \in G} \chi(a) = \sum_{a \in G} \chi(ga) = \sum_{a \in G} \chi(a),$$ meaning $\sum_{a \in G} \chi(a) = 0$, as desired. 
\end{proof}

\begin{prop} \label{3.3}
Suppose the group $G$ is abelian. Fix some $a \in G$. Then, $$\sum_{\chi \in \hat{G}} \chi(a) = \begin{cases}
|G| & \text{ if } a = 1 \\
0 & \text { otherwise } \\
\end{cases}$$ Here, $\hat{G} = \Hom(G, C^{\times})$ is the character group of $G$. 
\end{prop}

\begin{proof}
Using the fact that $G$ is noncanonically isomorphic to $\hat{G}$, this proof becomes identical to that of the previous proposition. 
\end{proof}

Before we introduce some new tools, let us provide some motivation to our treatment of L-functions. Let $K$ be a number field and $\mathfrak{m}$ be some modulus. Begin with the Dedekind zeta function, $\zeta_{K}(s)$. For some class $\mathfrak{t} \in C_{\mathfrak{m}}$ (i.e., the class group), define the partial zeta function to be $$\zeta(s, \mathfrak{t}) = \sum_{\mathfrak{a} \ge 0, \mathfrak{a} \in \mathfrak{t}} \frac{1}{N\mathfrak{a}^s}$$ Note that for every character $\chi$ of the class group, $$\zeta_{K}(s) = \sum_{\mathfrak{t} \in C_{\mathfrak{m}}} \zeta(s, \mathfrak{t}) \text{ and}$$ $$L(s, \chi) = \sum_{\mathfrak{t} \in C_{\mathfrak{m}}} \chi(\mathfrak{t})\zeta(s, \mathfrak{t})$$ In other words, knowing about $\zeta(s, \mathfrak{t})$ can tell us about the Dedekind zeta function as well as the corresponding L-function. 

\begin{thm}
The partial zeta function $\zeta(s, \mathfrak{t})$ is analytic for $\Re(s) > 1 - \frac{1}{[K : \QQ]}$ except for a simple pole at $s = 1$. If we let $g_{\mathfrak{m}}$ denote the residue at $s = 1$, then $g_{\mathfrak{m}}$ is independent of $\mathfrak{t}$.
\end{thm}

\begin{proof}
We omit the proof of this theorem, mainly because it relies on the famous class number formula. It allows us to determine exactly what $g_{\mathfrak{m}}$ is. 
\end{proof}

\begin{cor}
If $\chi$ is not the trivial character, the L-function $L(s, \chi)$ is analytic for $\Re(s) > 1 - \frac{1}{[K : \QQ]}$.
\end{cor}

\begin{proof}
Near $s = 1$, $$L(s, \chi) = \sum_{\mathfrak{t} \in C_{\mathfrak{m}}} \chi(\mathfrak{t})\zeta(s, \mathfrak{t}) = \frac{\sum_{\mathfrak{t} \in C_{\mathfrak{m}}} \chi(\mathfrak{t})g_{\mathfrak{m}}}{s - 1} + \text{ holomorphic function}$$ and Proposition \ref{3.2} shows us that the numerator of the first term is $0$.
\end{proof}


\subsection{Números Reais \(\RR\)}

At last, we come across one type of density. For a set $T$ of prime ideals of $K$, we define $\zeta_{K, T}(s) = \prod_{\mathfrak{p} \in T} \frac{1}{1 - N\mathfrak{p}^{-s}}$. 

\begin{defn}
If some positive integral power $\zeta_{K, T}(s)^{n}$ of $\zeta_{K, T}(s)$ extends to a meromorphic function on a neighborhood of $1$ having a pole of order $m$ at $1$, we say that $T$ has polar density $\delta(T) = \frac{m}{n}$. 
\end{defn}

\begin{prop}[Properties of Polar Density]

We have the following assertions:

\begin{enumerate}
    \item The set of all prime ideals of $K$ has polar density $1$.
    \item The polar density of every set is nonnegative.
    \item If $T$ is the disjoint union of $T_1$ and $T_2$, and two of the three polar densities exist, then so does the third, and we have $\delta(T) = \delta(T_1) + \delta(T_2)$.
    \item If $T \subset T'$, then $\delta(T) \leq \delta(T')$.
    \item A finite set has density zero. 
\end{enumerate}

\end{prop}

\begin{proof}

$ $\\ \vspace{-0.4cm}

\begin{enumerate}

    \item We know that $\zeta_{K, T}(s)$ extends to a neighborhood of $1$, where it has a simple pole. Thus $\frac{m}{n} = 1$, as desired.
    \item Having a negative polar density means $m < 0$, i.e., $\zeta_{K, T}(s)$ is holomorphic in a neighborhood of $s = 1$ and zero there. However, $\zeta_{K, T}(1) = \prod_{\mathfrak{p} \in T} \frac{1}{1 - N\mathfrak{p}^{-1}} > 0$, meaning polar density is nonnegative.
    \item Observe that $\zeta_{K, T}(s) = \zeta_{K, T_1}(s) \cdot \zeta_{K, T_2}(s)$. Suppose $\zeta_{K, T}(s)^{n}$ and $\zeta_{K, T_1}(s)^{n_1}$ extend to meromorphic functions with poles of order $m$ and $m_1$, respectively; the other two cases are identical. Then $$\zeta_{K, T_2}(s)^{nn_1} = \frac{\zeta_{K, T}(s)^{nn_1}}{\zeta_{K, T_1}(s)^{nn_1}}$$ extends to a meromorphic function in a neighborhood of $s = 1$ and has a pole there of order $mn_1 - m_1n$. Thus, $\delta(T_2) = \frac{mn_1 - m_1n}{nn_1} = \frac{m}{n} - \frac{m_1}{n_1} = \delta(T) - \delta(T_1)$, as desired.
    \item This follows readily from 3. 
    \item This is obvious; $m = 0$ because $\zeta_{K, T}(s)$ is finite and positive. Moreover, there is no pole at $s = 1$.
\end{enumerate}

\end{proof}

\begin{prop} \label{4.3}
If $T$ contains no primes $\mathfrak{p}$ for which $N\mathfrak{p}$ is prime (in $\ZZ$), then $\delta(T) = 0$. 
\end{prop}

\begin{proof}
Let $\mathfrak{p}$ be a prime in $T$. Since $N\mathfrak{p} = p^{f}$ (where $p$ lies under $\mathfrak{p}$ in $\ZZ$ and $f$ denotes the inertial degree of $\mathfrak{p}$), we must have $f \geq 2$; if $f = 1$, $N\mathfrak{p}$ would be prime. Moreover, for any given prime $p \in \ZZ$, there are at most $[K : \QQ]$ primes of $K$ lying over $p$. Thus, $\zeta_{K, T}(s)$ can be decomposed into a product $\prod_{1 \leq i \leq [K : \QQ]} g_{i}(s)$ of $d$ infinite products over the prime numbers, with each factor of $g_i$ being either a $1$ or a $\frac{1}{1-p^{-fs}}$ (for every prime $p$). Thus, for any $i$, $g_{i}(1) \leq \prod_{p} \frac{1}{1-p^{-f_{p}}} \leq \prod_{p} \frac{1}{1-p^{-2}} = \zeta(2) = \frac{\pi^2}{6}$. Thus, $g_{i}(s)$ is holomorphic at $s = 1$, meaning that the order of the pole there must be $0$ (recall that polar density cannot be negative). We conclude that $\delta(T) = 0$. 
\end{proof}

\begin{cor} \label{4.4}
Let $T_1$ and $T_2$ be sets of prime ideals in $K$. If the sets differ only by primes $\mathfrak{p}$ for which $N\mathfrak{p}$ is not prime and one of the two sets has polar density, then so does the other, and the densities are equal. 
\end{cor}

At last, the time has come to exploit the power of polar density. It turns out we can derive some important analytic results.

\begin{thm} \label{4.5}
Let $L \supset K$ be a field extension of finite degree and let $M$ be its Galois closure. Then the set of prime ideals of $K$ that split completely in $L$ has density $\frac{1}{[M : K]}$. 
\end{thm}

\begin{proof}
The first thing to notice is that a prime ideal $\mathfrak{p}$ of $K$ splits completely in $L$ if and only if it splits completely in $M$. One direction is easy: if it splits completely in $M$, it must split completely in the subfield $L$. If it splits completely in $L$, then it also splits completely in every conjugate field $L'$. All of these conjugate fields must lie under the decomposition field (the fixed field of the decomposition group of $\Gal(M/K)$), and so their compositum is a field lying under the decomposition field as well. This field is just $M$! $\mathfrak{p}$ splits completely only up to and including the decomposition field, so we conclude that it splits completely in $M$ as well. 

Thus, it suffices to prove this theorem with the assumption that $L$ is Galois over $K$. Let $S$ be the set of prime ideals of $K$ that split completely in $L$ and let $T$ be the primes of $L$ lying over a prime ideal in $S$. For each $\mathfrak{p} \in S$, there are exactly $[L : K]$ prime ideals $\mathfrak{P} \in T$, and for each of them, $N_{K}^{L}(\mathfrak{P}) = \mathfrak{p}$ (where $N_{K}^{L}$ denotes norm). Thus, $N\mathfrak{P} = N\mathfrak{p}$ (where $N$ denotes norm over $\QQ$). This tells us that $\zeta_{L, T}(s) = \zeta_{K, S}(s)^{[L : K]}$. Also, $T$ contains every prime ideal of $L$ that is unramified over $K$ and for which $N\mathfrak{P}$ is prime (in $\ZZ$). Thus, $T$ differs from the set of all prime ideals in $L$ by a set of polar density $0$ (using Corollary \ref{4.4}), and so $T$ has density $1$. Moreover, this shows that $\zeta_{K, S}$ has the property signifying that $S$ is a set of polar density $\frac{1}{[L : K]}$, as desired.
\end{proof}

\begin{cor}
If $f(x) \in K[x]$ splits into linear factors modulo $\mathfrak{p}$ for all but finitely many prime ideals $\mathfrak{p}$ of $K$, then $f$ splits into linear factors in $K$.
\end{cor}

\begin{proof}
If $L$ is the splitting field of $f$, then $L$ is Galois over $K$. Now, use Theorem \ref{4.5} on $L/K$. For more interesting details, see Bhandarkar\cite{bhandarkar2018hilbert}, Section 4. 
\end{proof}

\begin{cor}
For every abelian extension $L/K$ and every finite set $S$ of primes of $K$ including those that ramify in $L$, let $I_{K}^{S}$ denote the fractional ideals that are prime to all ideals in $S$. Then, the Artin map $$\left(\frac{L/K}{.}\right) : I_{K}^{S} \longrightarrow \Gal(L/K)$$ is surjective. 
\end{cor}

\begin{proof}
Let $H$ be the image of the Artin map; it is some subgroup of $\Gal(L/K)$. If its fixed field is $L^{H}$, then we see that $H = \Gal(L/L^{H})$ is the image. For all $\mathfrak{p} \not\in S$, $\left(\frac{L^{H}/K}{\mathfrak{p}}\right) = \left(\frac{L/K}{\mathfrak{p}}\right) \mid_{L^{H}} = 1$, which implies that $\mathfrak{p}$ splits completely in $L^{H}$. Thus, all but finitely many prime ideals of $\mathcal{O}_K$ split completely in $L^{H}$, so Theorem \ref{4.5} tells us that $[L^{H} : K] = 1$; in other words, the Artin map is surjective. 
\end{proof}


\section*{Bibliografia}
\begin{thebibliography}{1}

\bibitem{ref_tao1}
Tao, T.: \textit{Analysis I}. 1st ed. Hindustan Book Agency (2006)
\end{thebibliography}


\end{document}
