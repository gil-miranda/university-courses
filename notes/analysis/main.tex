\documentclass{article}
%\documentclass[runningheads]{llncs}
%
%\usepackage{graphicx}
% Used for displaying a sample figure. If possible, figure files should
% be included in EPS format.
%
% If you use the hyperref package, please uncomment the following line
% to display URLs in blue roman font according to Springer's eBook style:
% \renewcommand\UrlFont{\color{blue}\rmfamily}
%\documentclass[12pt]{amsart}
\RequirePackage{etex}

\usepackage[margin=1in]{geometry}
\usepackage[english]{babel}
\usepackage[utf8]{inputenc}
\usepackage{subcaption}
\usepackage{amsmath}
\usepackage{amssymb}
\usepackage{amsfonts}
\usepackage{amsthm}
\usepackage{mathrsfs}
%\usepackage[all]{xy}
\usepackage[pdftex]{graphicx}
\usepackage{color}
\usepackage{cite}
\usepackage{url}
\usepackage{indent first}
\usepackage[labelfont=bf,labelsep=period,justification=raggedright]{caption}
\usepackage[english]{babel}
\usepackage[utf8]{inputenc}
\usepackage[colorlinks=true,linkcolor=blue]{hyperref}
\usepackage[colorinlistoftodos]{todonotes}
%\usepackage{tkz-fct}
%\usepackage{tikz}
%\usetikzlibrary{calc}
%\usepackage{multicol}
%\PassOptionsToPackage{dvipsnames,svgnames}{xcolor}
%\usepackage{textcomp}


% \topmargin 0.4cm
% \oddsidemargin 0.5cm
% \evensidemargin 0.5cm
% \textwidth 14cm 
% \textheight 20.2cm

% \setlength{\oddsidemargin}{0.25in}
% \setlength{\evensidemargin}{0.25in}
% \setlength{\textwidth}{6in}
%\setlength{\topmargin}{-0.25in}
%\setlength{\textheight}{8in}

\DeclareMathOperator{\ab}{ab}
%\DeclareMathOperator{\arg}{arg}
\DeclareMathOperator{\Aut}{Aut}
\DeclareMathOperator{\BGL}{BGL}
\DeclareMathOperator{\Br}{Br}
\DeclareMathOperator{\card}{card}
\DeclareMathOperator{\ch}{ch}
\DeclareMathOperator{\Char}{char}
\DeclareMathOperator{\CHur}{CHur}
\DeclareMathOperator{\Cl}{Cl}
\DeclareMathOperator{\coker}{coker}
\DeclareMathOperator{\Conf}{Conf}
\DeclareMathOperator{\disc}{disc}
\DeclareMathOperator{\End}{End}
\DeclareMathOperator{\et}{\text{\'et}}
\DeclareMathOperator{\Fix}{Fix}
\DeclareMathOperator{\Gal}{Gal}
\DeclareMathOperator{\GL}{GL}
\DeclareMathOperator{\Hom}{Hom}
\DeclareMathOperator{\Hur}{Hur}
\DeclareMathOperator{\im}{im}
\DeclareMathOperator{\Ind}{Ind}
\DeclareMathOperator{\Inn}{Inn}
\DeclareMathOperator{\Irr}{Irr}
\DeclareMathOperator{\lcm}{lcm}
\DeclareMathOperator{\Mor}{Mor}
\DeclareMathOperator{\ord}{ord}
\DeclareMathOperator{\Out}{Out}
\DeclareMathOperator{\Perm}{Perm}
\DeclareMathOperator{\PGL}{PGL}
\DeclareMathOperator{\Pin}{Pin}
\DeclareMathOperator{\PSL}{PSL}
\DeclareMathOperator{\rad}{rad}
%\DeclareMathOperator{\Re}{Re}
\DeclareMathOperator{\SL}{SL}
\DeclareMathOperator{\SO}{SO}
\DeclareMathOperator{\Spec}{Spec}
\DeclareMathOperator{\Spin}{Spin}
\DeclareMathOperator{\St}{St}
\DeclareMathOperator{\Surj}{Surj}
\DeclareMathOperator{\Syl}{Syl}
\DeclareMathOperator{\tame}{tame}
\DeclareMathOperator{\Tr}{Tr}
\DeclareMathOperator{\fancyC}{Č}

\newcommand{\eps}{\varepsilon}
\newcommand{\QED}{\hspace{\stretch{1}} $\blacksquare$}
\renewcommand{\AA}{\mathbb{A}}
\newcommand{\CC}{\mathbb{C}}
\newcommand{\EE}{\mathbb{E}}
\newcommand{\FF}{\mathbb{F}}
\newcommand{\HH}{\mathbb{H}}
\newcommand{\NN}{\mathbb{N}}
\newcommand{\OO}{\mathbb{O}}
\newcommand{\PP}{\mathbb{P}}
\newcommand{\QQ}{\mathbb{Q}}
\newcommand{\RR}{\mathbb{R}}
\newcommand{\ZZ}{\mathbb{Z}}
\newcommand{\bfm}{\mathbf{m}}
\newcommand{\mcA}{\mathcal{A}}
\newcommand{\mcG}{\mathcal{G}}
\newcommand{\mcH}{\mathcal{H}}
\newcommand{\mcM}{\mathcal{M}}
\newcommand{\mcN}{\mathcal{N}}
\newcommand{\mcO}{\mathcal{O}}
\newcommand{\mcP}{\mathcal{P}}
\newcommand{\mcQ}{\mathcal{Q}}
\newcommand{\mfa}{\mathfrak{a}}
\newcommand{\mfb}{\mathfrak{b}}
\newcommand{\mfc}{\mathfrak{c}}
\newcommand{\mfI}{\mathfrak{I}}
\newcommand{\mfM}{\mathfrak{M}}
\newcommand{\mfm}{\mathfrak{m}}
\newcommand{\mfo}{\mathfrak{o}}
\newcommand{\mfO}{\mathfrak{O}}
\newcommand{\mfP}{\mathfrak{P}}
\newcommand{\mfp}{\mathfrak{p}}
\newcommand{\mfq}{\mathfrak{q}}
\newcommand{\mfz}{\mathfrak{z}}
\newcommand{\msP}{\mathscr{P}}
\newcommand{\AGL}{\mathbb{A}\GL}
\newcommand{\Qbar}{\overline{\QQ}}
\renewcommand{\qedsymbol}{$\blacksquare$}

\DeclareRobustCommand{\sstirling}{\genfrac\{\}{0pt}{}}
\DeclareRobustCommand{\fstirling}{\genfrac[]{0pt}{}}
\def\multiset#1#2{\ensuremath{\left(\kern-.3em\left(\genfrac{}{}{0pt}{}{#1}{#2}\right)\kern-.3em\right)}}

\newcommand{\planefig}[2] {
\filldraw[shift={#1},rotate=#2] (.4,.3) circle (2pt);
\filldraw[shift={#1},rotate=#2] (-.4,.3) circle (2pt);
\draw[shift={#1},rotate=#2] (-20:.55) arc (-20:-160:.55);
\draw[shift={#1},rotate=#2] (0,0) circle (1cm);
}

\newcommand{\simon}[1]{\todo[color=green]{SR: #1}}
\newcommand{\nitya}[1]{\todo[color=blue!30]{NM: #1}}
\newcommand{\penghui}[1]{\todo[color=red!60]{HPH: #1}}
\newcommand{\michael}[1]{\todo[color=yellow]{MW: #1}}
\newcommand{\winnie}[1]{\todo[color=purple!60]{WL: #1}}
\newcommand{\peter}[1]{\todo[color=pink]{PR: #1}}
\newcommand{\jae}[1]{\todo[color=brown!60]{JL: #1}}
\newcommand{\silas}[1]{\todo[color=orange]{SJ: #1}}
\newcommand{\refr}[1]{\textcolor{blue}{#1}}


\theoremstyle{plain}
\newtheorem{thm}{Teorema}
%\newtheorem{lemma}[thm]{Lemma}
\newtheorem{cor}{Corolário}
\newtheorem{conj}{Conjectura}
\newtheorem{prop}{Proposição}
\newtheorem*{prop*}{Proposição}
\newtheorem{lemma}{Lema}
\newtheorem*{lemma*}{Lema}
\newtheorem{heur}{Heuristica}
\newtheorem{qn}{Questão}
%\newtheorem{claim}[thm]{Claim}
\newtheorem{axm}{Axioma}
\newtheorem{defn}{Definição}
\newtheorem{cond}{Condições}
\newtheorem*{notn}{Notação}

\theoremstyle{remark}
\newtheorem{rem}{Remark}
\newtheorem*{ex}{Exemplo}
\newtheorem*{exer}{Exercicio}

\numberwithin{equation}{section}
\numberwithin{thm}{section}
\numberwithin{defn}{section}
\numberwithin{lemma}{section}
\numberwithin{axm}{section}

\usepackage{arxiv}

\usepackage[utf8]{inputenc} % allow utf-8 input
\usepackage[T1]{fontenc}    % use 8-bit T1 fonts
\usepackage{hyperref}       % hyperlinks
\usepackage{url}            % simple URL typesetting
\usepackage{booktabs}       % professional-quality tables
\usepackage{amsfonts}       % blackboard math symbols
%\usepackage{nicefrac}       % compact symbols for 1/2, etc.
\usepackage{microtype}      % microtypography
%\usepackage{lipsum}

\title{Notas de estudo em Análise I (Análise Real)\\ Um guia de teoremas, resultados importantes\\ e exercícios}


\author{
  Gil S. M. Neto\\
  Graduando em Matemática Aplicada - UFRJ\\
  \texttt{gilsmneto@gmail.com, gil.neto@ufrj.br}\\
  \texttt{http://mirandagil.github.io}
  %% examples of more authors
  %% \AND
  %% Coauthor \\
  %% Affiliation \\
  %% Address \\
  %% \texttt{email} \\
  %% \And
  %% Coauthor \\
  %% Affiliation \\
  %% Address \\
  %% \texttt{email} \\
  %% \And
  %% Coauthor \\
  %% Affiliation \\
  %% Address \\
  %% \texttt{email} \\
}
\date{Última atualização: \today}

\begin{document}

\maketitle

\tableofcontents
\newpage

\section{Teoria Ingênua dos Conjuntos}

\begin{defn}[Informal de conjuntos]\phantomsection\label{def1:1}
Um conjunto é uma coleção não ordenada de objetos. Se \(x\) é um objeto do conjunto \(A\), dizemos \(x \in A\), caso contrário dizemos \(x \not\in A\). \\
Exemplo: \(3 \in \{1,2,3,4,5\}; \, \, 7 \not\in \{1,2,3,4,5\}\)
\end{defn}

	\begin{axm} [Conjuntos são objetos]\phantomsection\label{axm1:1}
		Se \(A\) é um conjunto, então \(A\) também é um objeto, ou seja, se existe outro conjunto \(B\), então faz sentido inferir \(A \in B\) ou \(A \not\in B\)
\end{axm}

\begin{ex}
Seja \(B = \{ 1, 3, \{4, 5\}, 8 \}; \,\, A = \{4,5\}\), então \(A \in B\)\\
Seja \(C = \{ 1, 3, 4, 5, 8 \}; \,\, D = \{4,5\}\), então \(C \subset D\)\\
é importante notar que apesar de \(4 \in A, 5 \in A\), é verdade que \(4 \not\in B, 5 \not\in B\) (verificar)
\end{ex}

\begin{defn}[Subconjuntos]\phantomsection\label{def1:2}
  \(A \subset B \iff x \in A \implies x \in B, \, \forall x \in A\)
\end{defn}

\begin{defn}[Igualdade de Conjuntos] \phantomsection\label{def1:3}
Definimos dois conjuntos \(A = B \iff A \subset B \wedge B \subset A\)\\
Ou seja, \(x \in A \implies x \in B, \, \forall x \in A \wedge y \in B \implies y \in A, \, \forall y \in B\)
\end{defn}

 \begin{axm}[Conjunto Vazio]\phantomsection\label{axm1:2}
 Existe um conjunto ao qual nenhum objeto pertence. A este grupo denominamos \(\emptyset\) .\\
 Para qualquer objeto \(x\), temos \(x \not\in \emptyset\).\\

\begin{lemma}[O Conjunto vazio é subconjunto de todo conjunto]\phantomsection\label{lem1:1}
 Seja \(A\) um conjunto qualquer, então \(\emptyset \subset A\)\\
 \begin{proof}
 Suponha que \(\emptyset \not\subset A\), para qualquer conjunto \(A\). Para negar a Definição \ref{def1:2} teremos: \(A \not\subset B \iff \exists \, x \in A; x \not\in B\) \\
 Logo, para termos \(\emptyset \not\subset A\), deve existir um objeto em \(\emptyset\) que não está contido em \(A\), mas não há nenhum objeto em \(\emptyset\), logo uma contradição, e temos \(\emptyset \subset A, \, \forall A\)
 \end{proof}
\end{lemma} 
 
 \begin{lemma}[O conjunto vazio é único]\phantomsection\label{lem1:2}
 	\begin{proof}
			 Seja \(\emptyset, \emptyset'\) conjuntos vazios, então do Lema \ref{lem1:1} temos \(\emptyset \subset 	\emptyset'\) e \(\emptyset' \subset \emptyset\), e pela Definição \ref{def1:3} \(\emptyset = \emptyset'\).
	 \end{proof}
 \end{lemma}
 \end{axm}
 
 \begin{lemma}[Escolha única]\phantomsection\label{lem1:3}
 Seja \(A\) um conjunto não vazio, então existe ao menos um \(x\) tal que \(x \in A\)\\
 \begin{proof}
 Suponha que não exista nenhum objeto x pertencente a \(A\), então: \(x \not\in A, \, \forall x\), mas isso implicaria que A é um conjunto vazio, o que contraria a hipótese.
 \end{proof}
\end{lemma}
Este lema nos permite escolher algum elemento de A. Ainda mais, dado uma família finita de Conjuntos \(A_1, A_2, \dots, A_n \), podemos escolher um elemento de cada conjunto \(x_1, x_2, \dots, x_n\). Para o caso infinito cairá no Axioma da Escolha, assunto a ser desenvolvido em outro momento.

\begin{axm}[Singleton]\phantomsection\label{axm1:3}
Dado um objeto \(a\), então existe um conjunto  de apenas um elemento \(\{a\}\). Ou seja, para todo objeto \(x, x \in \{a\} \iff y = a\). Ainda mais, para todo objeto \(a, b\) existe um conjunto \(\{a,b\}\) onde \(\forall y, y \in \{a,b\} \iff y = a \vee y = b\) 

\end{axm}

\section{A construção dos números}
\subsection{Números Naturais e Inteiros \(\NN, \ZZ\)}

\subsection{Números Racionais \(\QQ\) }

\subsection{Números Reais \(\RR\)}

\section{Demonstrações sobre os Reais}
\subsection{\(\sqrt{2} \notin \QQ\)}
Vamos mostrar que a equação
\[
p^2 = 2 \tag{1} \label{eq}
\]
não tem solução nos racionais.\\
Primeiro vamos provar uma pequena proposição 
\begin{prop*}[\(n^2\) par \(\iff n\) é par].\\
\begin{proof} \(\Leftarrow\)\\
\(n\) par implica que podemos escrever \(n = 2m\) para um dado \(m\). Então
\[n^2 = (2m)^2 = 4m^2\]
e podemos reescrever \(4m^2 = 2\cdot(2m^2)\), logo \(n^2\) é par. \\
\(\Rightarrow\)\\
Vamos assumir que \(n\) seja ímpar, então podemos escrever \(n = 2m+1\), logo
\[
n^2 = (2m+1)^2 = (4m^2 + 4m + 1) = 2(2m^2 + 2m) +1
\]
O que nos daria que \(n^2\) é ímpar contrariando a hipótese, logo \(n\) é par.
\end{proof}
\end{prop*}

Assumindo por hipótese que \eqref{eq} tenha solução nos reais, podemos escrever \(p = \frac{m}{n}\) com \(mdc(m,n) = 1\) , temos então
\begin{align*}
\frac{m^2}{n^2} &= 2\\
m^2 &= 2n^2
\end{align*}
Mas isso nos dá \(m^2\) que pela proposição implica em \(m\) par. Mas se \(m^2\) é par, então \(m^2 \geq 4\), o que nos diz que \(2n^2\) é divísivel por \(4\), com uma manipulação chegamos que \(n^2\) par implicando que \(n\) é par, então \(mdc(n,m) \neq 1\) o que contraria a hipótese, logo \(\sqrt{2} \notin \QQ\) 


\section*{Bibliografia}
\begin{thebibliography}{1}

\bibitem{ref_tao1}
Tao, T.: \textit{Analysis I}. 1st ed. Hindustan Book Agency (2006)

\bibitem{ref_rudin1}
Rudin, W.: \textit{Principles of Mathematical Analysis}. 3rd ed. McGraw-Hill (1976)

\bibitem{ref_lima1}
Lima, E.: \textit{Curso de Análise vol I}. 14ª ed. IMPA (2016)

\bibitem{ref_cabral}
Neri, C. \& Cabral, M.: \textit{Curso de Análise Real}. 2ª ed. (2011)
\end{thebibliography}


\end{document}
