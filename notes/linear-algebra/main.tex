\documentclass{article}
%\documentclass[runningheads]{llncs}
%
%\usepackage{graphicx}
% Used for displaying a sample figure. If possible, figure files should
% be included in EPS format.
%
% If you use the hyperref package, please uncomment the following line
% to display URLs in blue roman font according to Springer's eBook style:
% \renewcommand\UrlFont{\color{blue}\rmfamily}
%\documentclass[12pt]{amsart}
\RequirePackage{etex}

\usepackage[margin=1in]{geometry}
\usepackage[english]{babel}
\usepackage[utf8]{inputenc}
\usepackage{subcaption}
\usepackage{amsmath}
\usepackage{amssymb}
\usepackage{amsfonts}
\usepackage{amsthm}
\usepackage{mathrsfs}
%\usepackage[all]{xy}
\usepackage[pdftex]{graphicx}
\usepackage{color}
\usepackage{cite}
\usepackage{url}
\usepackage{indent first}
\usepackage[labelfont=bf,labelsep=period,justification=raggedright]{caption}
\usepackage[english]{babel}
\usepackage[utf8]{inputenc}
\usepackage[colorlinks=true,linkcolor=blue]{hyperref}
\usepackage[colorinlistoftodos]{todonotes}
%\usepackage{tkz-fct}
\usepackage{tikz}
\usetikzlibrary{calc}
\usepackage{multicol}
\PassOptionsToPackage{dvipsnames,svgnames}{xcolor}
\usepackage{textcomp}


% \topmargin 0.4cm
% \oddsidemargin 0.5cm
% \evensidemargin 0.5cm
% \textwidth 14cm 
% \textheight 20.2cm

% \setlength{\oddsidemargin}{0.25in}
% \setlength{\evensidemargin}{0.25in}
% \setlength{\textwidth}{6in}
%\setlength{\topmargin}{-0.25in}
%\setlength{\textheight}{8in}

\DeclareMathOperator{\ab}{ab}
%\DeclareMathOperator{\arg}{arg}
\DeclareMathOperator{\Aut}{Aut}
\DeclareMathOperator{\BGL}{BGL}
\DeclareMathOperator{\Br}{Br}
\DeclareMathOperator{\card}{card}
\DeclareMathOperator{\ch}{ch}
\DeclareMathOperator{\Char}{char}
\DeclareMathOperator{\CHur}{CHur}
\DeclareMathOperator{\Cl}{Cl}
\DeclareMathOperator{\coker}{coker}
\DeclareMathOperator{\Conf}{Conf}
\DeclareMathOperator{\disc}{disc}
\DeclareMathOperator{\End}{End}
\DeclareMathOperator{\et}{\text{\'et}}
\DeclareMathOperator{\Fix}{Fix}
\DeclareMathOperator{\Gal}{Gal}
\DeclareMathOperator{\GL}{GL}
\DeclareMathOperator{\Hom}{Hom}
\DeclareMathOperator{\Hur}{Hur}
\DeclareMathOperator{\im}{im}
\DeclareMathOperator{\Ind}{Ind}
\DeclareMathOperator{\Inn}{Inn}
\DeclareMathOperator{\Irr}{Irr}
\DeclareMathOperator{\lcm}{lcm}
\DeclareMathOperator{\Mor}{Mor}
\DeclareMathOperator{\ord}{ord}
\DeclareMathOperator{\Out}{Out}
\DeclareMathOperator{\Perm}{Perm}
\DeclareMathOperator{\PGL}{PGL}
\DeclareMathOperator{\Pin}{Pin}
\DeclareMathOperator{\PSL}{PSL}
\DeclareMathOperator{\rad}{rad}
%\DeclareMathOperator{\Re}{Re}
\DeclareMathOperator{\SL}{SL}
\DeclareMathOperator{\SO}{SO}
\DeclareMathOperator{\Spec}{Spec}
\DeclareMathOperator{\Spin}{Spin}
\DeclareMathOperator{\St}{St}
\DeclareMathOperator{\Surj}{Surj}
\DeclareMathOperator{\Syl}{Syl}
\DeclareMathOperator{\tame}{tame}
\DeclareMathOperator{\Tr}{Tr}
\DeclareMathOperator{\fancyC}{Č}

\newcommand{\eps}{\varepsilon}
\newcommand{\QED}{\hspace{\stretch{1}} $\blacksquare$}
\renewcommand{\AA}{\mathbb{A}}
\newcommand{\CC}{\mathbb{C}}
\newcommand{\EE}{\mathbb{E}}
\newcommand{\FF}{\mathbb{F}}
\newcommand{\HH}{\mathbb{H}}
\newcommand{\NN}{\mathbb{N}}
\newcommand{\OO}{\mathbb{O}}
\newcommand{\PP}{\mathbb{P}}
\newcommand{\QQ}{\mathbb{Q}}
\newcommand{\RR}{\mathbb{R}}
\newcommand{\ZZ}{\mathbb{Z}}
\newcommand{\bfm}{\mathbf{m}}
\newcommand{\mcA}{\mathcal{A}}
\newcommand{\mcG}{\mathcal{G}}
\newcommand{\mcH}{\mathcal{H}}
\newcommand{\mcM}{\mathcal{M}}
\newcommand{\mcN}{\mathcal{N}}
\newcommand{\mcO}{\mathcal{O}}
\newcommand{\mcP}{\mathcal{P}}
\newcommand{\mcQ}{\mathcal{Q}}
\newcommand{\mfa}{\mathfrak{a}}
\newcommand{\mfb}{\mathfrak{b}}
\newcommand{\mfc}{\mathfrak{c}}
\newcommand{\mfI}{\mathfrak{I}}
\newcommand{\mfM}{\mathfrak{M}}
\newcommand{\mfm}{\mathfrak{m}}
\newcommand{\mfo}{\mathfrak{o}}
\newcommand{\mfO}{\mathfrak{O}}
\newcommand{\mfP}{\mathfrak{P}}
\newcommand{\mfp}{\mathfrak{p}}
\newcommand{\mfq}{\mathfrak{q}}
\newcommand{\mfz}{\mathfrak{z}}
\newcommand{\msP}{\mathscr{P}}
\newcommand{\AGL}{\mathbb{A}\GL}
\newcommand{\Qbar}{\overline{\QQ}}
\renewcommand{\qedsymbol}{$\blacksquare$}

\DeclareRobustCommand{\sstirling}{\genfrac\{\}{0pt}{}}
\DeclareRobustCommand{\fstirling}{\genfrac[]{0pt}{}}
\def\multiset#1#2{\ensuremath{\left(\kern-.3em\left(\genfrac{}{}{0pt}{}{#1}{#2}\right)\kern-.3em\right)}}

\newcommand{\planefig}[2] {
\filldraw[shift={#1},rotate=#2] (.4,.3) circle (2pt);
\filldraw[shift={#1},rotate=#2] (-.4,.3) circle (2pt);
\draw[shift={#1},rotate=#2] (-20:.55) arc (-20:-160:.55);
\draw[shift={#1},rotate=#2] (0,0) circle (1cm);
}

\newcommand{\simon}[1]{\todo[color=green]{SR: #1}}
\newcommand{\nitya}[1]{\todo[color=blue!30]{NM: #1}}
\newcommand{\penghui}[1]{\todo[color=red!60]{HPH: #1}}
\newcommand{\michael}[1]{\todo[color=yellow]{MW: #1}}
\newcommand{\winnie}[1]{\todo[color=purple!60]{WL: #1}}
\newcommand{\peter}[1]{\todo[color=pink]{PR: #1}}
\newcommand{\jae}[1]{\todo[color=brown!60]{JL: #1}}
\newcommand{\silas}[1]{\todo[color=orange]{SJ: #1}}
\newcommand{\refr}[1]{\textcolor{blue}{#1}}


\theoremstyle{plain}
\newtheorem{thm}{Teorema}
%\newtheorem{lemma}[thm]{Lemma}
\newtheorem{cor}{Corolário}
\newtheorem{conj}{Conjectura}
\newtheorem{prop}{Proposição}
\newtheorem{lemma}{Lema}
\newtheorem{heur}{Heuristica}
\newtheorem{qn}{Questão}
%\newtheorem{claim}[thm]{Claim}
\newtheorem{axm}{Axioma}
\newtheorem{defn}{Definição}
\newtheorem{cond}{Condições}
\newtheorem*{notn}{Notação}

\theoremstyle{remark}
\newtheorem{rem}{Remark}
\newtheorem*{ex}{Exemplo}
\newtheorem*{exer}{Exercicio}

\numberwithin{equation}{section}
\numberwithin{thm}{section}
\numberwithin{defn}{section}
\numberwithin{lemma}{section}
\numberwithin{axm}{section}

\usepackage{arxiv}

\usepackage[utf8]{inputenc} % allow utf-8 input
\usepackage[T1]{fontenc}    % use 8-bit T1 fonts
\usepackage{hyperref}       % hyperlinks
\usepackage{url}            % simple URL typesetting
\usepackage{booktabs}       % professional-quality tables
\usepackage{amsfonts}       % blackboard math symbols
%\usepackage{nicefrac}       % compact symbols for 1/2, etc.
\usepackage{microtype}      % microtypography
%\usepackage{lipsum}

\title{Notas de estudo em Álgebra Linear\\ Um guia de teoremas, resultados importantes\\ e exercícios}


\author{
  Gil S. M. Neto\\
  Graduando em Matemática Aplicada - UFRJ\\
  \texttt{gilsmneto@gmail.com, gil.neto@ufrj.br}\\
  \texttt{http://mirandagil.github.io}
  %% examples of more authors
  %% \AND
  %% Coauthor \\
  %% Affiliation \\
  %% Address \\
  %% \texttt{email} \\
  %% \And
  %% Coauthor \\
  %% Affiliation \\
  %% Address \\
  %% \texttt{email} \\
  %% \And
  %% Coauthor \\
  %% Affiliation \\
  %% Address \\
  %% \texttt{email} \\
}
\date{Última atualização: \today}

\begin{document}

\maketitle

\tableofcontents
\newpage

\section{Fundamentos}

\subsection{O Espaço Vetorial}

Um espaço vetorial \(\mathbb{V}\) é um objeto matemático, dizemos \( \mathbb{V}\) é um E.V. sobre um corpo \(\mathbb{K}\), onde são definidas duas operações:

\begin{itemize}
	\item Soma (+): \(u+v = z \, \,\forall  u,v, z \in \mathbb{V}\)
	\item Multiplicação por escalar (\(\cdot\)): \(k\cdot v = z \, \, \, \forall k \in \mathbb{K}, v,z \in \mathbb{V}\)
\end{itemize}

\subsubsection*{Axiomas de Espaços Vetoriais}
\begin{enumerate}
\item Soma é comutativa: \(u+v = v+u\)
\item Soma é associativa \( u+(v+z) = (u+v)+z\)
\item Existe um elemento  \(E \in \mathbb{V}\), chamando elemento neutro tal que \(v + E = v, \forall \, v \in \mathbb{V}\)
\item Para todo \(v \in \mathbb{V} \,\,\, \exists -v\), tal que: \(v + (-v) = E\)
\item Multiplicação por escalar é associativa: \( a(b u) = (ab) u \, \, \forall \, \,  a, b \in \mathbb{K}, u \in \mathbb{V}\)
\item Multiplicação por escalar é distributiva: \( (a+b)u = au + bu \) e \(a(u+v) = au + av\)
\item \( 1 \cdot v = v, \forall \, \, v \in \mathbb{V} \)
\end{enumerate}

\subsubsection*{Consequência dos Axiomas}
\begin{itemize}
	\item O Elemento neutro é único  \begin{proof}
	  		Vamos supor que exista outro  elemento neutro: \(E \text{ e } E'\) logo: \( E + v = v = E' + v\)\\
	  		utilizando o inverso da soma
	  		\begin{align*}
	  		E + v &= E' + v \\
	  		E + v - v &= E' + v - v \\
	  		E + E &= E ' + E' \\
	  		E &= E'
	  		\end{align*}
					  		
	  		
	  \end{proof}

	\item Lei do cancelamento: \(w + v = w + u \implies v = u, \forall \, w, v, u \in \mathbb{V}\)
	\begin{proof}
		\begin{align*}
			w + v &= w + u\\
			&\text{Utilizando existência do inverso}\\
			w + v - w &= w + u - w\\
			v + E &= u + E\\
			&\text{Utilizando o elemento neutro}\\
			v = u
		\end{align*}		
		\end{proof}
		\item \( 0\cdot v = E \, \, \, \forall \,  v \in \mathbb{V}\)
	\begin{proof}
		\begin{align*}
			&\text{Usando distributividade}\\
			(0+0) v &= 0v \\
			0v + 0v &= 0v \\
			&\text{Usando existência do inverso}\\
			0v + 0v - 0v &= 0v - 0v \\
			0v &= E
		\end{align*}		
	\end{proof}
			\item \( a \neq 0, v \neq E \implies av \neq E, \, \, \forall \,  v \in \mathbb{V}, a \in \mathbb{K}\)
	\begin{proof}
		\begin{align*}
			av &= E \\
			1v &= v\\
			&= (a \cdot a^{-1}) v\\
			&= a^{-1}(av) \\
			&=a^{-1} E\\
			v &= E
		\end{align*}		
	\end{proof}
	\item \(-1 \cdot v = -v, \forall \, v \in \mathbb{V}\)
	\begin{proof}
		\begin{align*}
			v + -1v &= 1v + -1v\\
			\text{Utilizando distributividade da multiplicação}
			v + -1v &= (1 - 1) v\\
			v + -1v &= 0v\\
			\text{Propriedade de elemento neutro}\\
			v + -1v &= E\\
			v - v + -1v &= E -v\\
			E + -1v &= -v\\
			-1v &= -v			
		\end{align*}
	\end{proof}
		\item E = (0,0,\dots ,0)
		\begin{proof}
			\( z = (0,0,\dots ,0), v = (v_1, \dots , v_n) \)
			\begin{align*}
				v + z &= (v_1 + 0, v_2 + 0, \dots , v_n + 0)\\
				v + z &= (v_1, v_2, \dots, v_n)\\
				v + z &= v\\
				&\text{z tem então propriedade de elemento neutro, mas o elemento neutro é único, logo}\\
				z &= E 
			\end{align*}
		\end{proof}
\end{itemize}

\subsection{Isomorfismo}
Isomorfismo é um mapa que preserva as estruturas de grupo entre dois conjuntos.\\
Dois espaços lineares são isomorfos se são homeomorfos e se há uma bijeção entre eles. Espaços lineares isomorfos são indistiguiveis pelas operações definidas neles.\\
Se temos dois espaços vetoriais isomorfos \(\mathbb{V}, \mathbb{U}\) com \(a,b,c \in \mathbb{V}\) e \( x,y,z \in \mathbb{U}\) e temos \(a = x, b = y e c = z\), então:
\[ a + b = c \implies x + y = z\]


\section*{Bibliografia}
\begin{thebibliography}{1}

\bibitem{ref_tao1}
Peter D. Lax: \textit{Linear Algebra and it's applications}. 2nd ed. Wiley (2007)
\end{thebibliography}


\end{document}
